% XeLaTeX can use any Mac OS X font. See the setromanfont command below.
% Input to XeLaTeX is full Unicode, so Unicode characters can be typed directly into the source.

% The next lines tell TeXShop to typeset with xelatex, and to open and save the source with Unicode encoding.

%!TEX TS-program = xelatex
%!TEX encoding = UTF-8 Unicode

%\documentclass[12pt]{article}
\documentclass[a4paper]{article}
\usepackage{geometry}                	% See geometry.pdf to learn the layout options. There are lots.
\geometry{letterpaper}                  % ... or a4paper or a5paper or ... 
%\geometry{landscape}                	% Activate for for rotated page geometry
%\usepackage[parfill]{parskip}    		% Activate to begin paragraphs with an empty line rather than an indent
\usepackage{graphicx}
\usepackage{amssymb}

%page dimension
%\hoffset = 0.5in
%\voffset = 0.5in

% Will Robertson's fontspec.sty can be used to simplify font choices.
% To experiment, open /Applications/Font Book to examine the fonts provided on Mac OS X,
% and change "Hoefler Text" to any of these choices.

\usepackage{fontspec,xltxtra,xunicode}
\defaultfontfeatures{Mapping=tex-text}
%\setromanfont[Mapping=tex-text]{Hoefler Text}
%\setsansfont[Scale=MatchLowercase,Mapping=tex-text]{Gill Sans}
%\setmonofont[Scale=MatchLowercase]{Andale Mono}
\setmainfont[Scale=1.1]{TH Sarabun New} 
\XeTeXlinebreaklocale 'th_TH'

% add a dot after the section number
\usepackage{titlesec}
\titlelabel{\thetitle.\quad}

\titleformat*{\section}{\Large\bfseries}
\titleformat*{\subsection}{\Large}
\titleformat*{\subsubsection}{\Large}
\titleformat*{\paragraph}{\large\bfseries}
\titleformat*{\subparagraph}{\large\bfseries}

% add tab
\newcommand{\tab}{\hspace{1.27cm}}

\reversemarginpar % Move the margin to the left of the page 
\newcommand{\MarginText}[1]{\marginpar{\raggedleft\itshape\small#1}} % New command defining the margin text style
%\newcommand{\MarginText}[1]{\marginpar{\raggedleft#1}} % New command defining the margin text style
\newcommand{\Description}[1]{\hangindent=2em\hangafter=0\noindent\raggedright\footnotesize{#1}\par\normalsize\vspace{1em}}
%\newcommand{\Description}[1]{\hangindent=1.27cm\hangafter=0\noindent\raggedright{#1}} % Define a command for descriptions of each entry - change spacing and font sizes here

\title{ข้อเสนอโครงงานมหาบัณฑิต (MASTER PROJECT PROPOSAL)}
\author{The Author}
%\date{}								% Activate to display a given date or no date

%Paragraph indent and break
\usepackage{indentfirst}
\setlength{\parindent}{1.27cm}
\setlength{\parskip}{1ex plus 0.5ex minus 0.2ex}
%\setlength{\parskip}{1cm plus4mm minus3mm}

% For many users, the previous commands will be enough.
% If you want to directly input Unicode, add an Input Menu or Keyboard to the menu bar 
% using the International Panel in System Preferences.
% Unicode must be typeset using a font containing the appropriate characters.
% Remove the comment signs below for examples.

% \newfontfamily{\A}{Geeza Pro}
% \newfontfamily{\H}[Scale=0.9]{Lucida Grande}
% \newfontfamily{\J}[Scale=0.85]{Osaka}
%----------------------------------------------------------------------------------------
\begin{document}
%\maketitle

\begin{center}
{\huge \bf ข้อเสนอโครงงานมหาบัณฑิต} 
\end{center}

\begin{center}
{\huge \bf (MASTER PROJECT PROPOSAL)} 
\end{center}

\vspace{1cm}
\Large{\noindent\hspace{0.7cm}\setlength{\tabcolsep}{15pt}
\begin{tabular}{l l}    
	\bf ชื่อเรื่อง (ภาษาไทย) 		& การพัฒนาระบบสำหรับการรวบรวมบัตรสมาชิกบนเทคโนโลยี \\
							& เนียร์ฟิลด์คอมมูนิเคชันด้วยแอนดรอยด์แพลตฟอร์ม \\
	\bf ชื่อเรื่อง (ภาษาอังกฤษ)		& Developing a NFC Based Integrated Member Card System \\
							& for Mobile Devices Using the Android Platform \\ 
							& \\
							& \\
	\bf เสนอโดย				& นายณัฐพล แซ่ลิ้ม \\
	\bf เลขประจำตัวนิสิต			& 557 09752 21 \\
	\bf สาขาวิชา				& วิศวกรรมซอฟต์แวร์ \\
	\bf ภาควิชา				& วิศวกรรมคอมพิวเตอร์ \\
	\bf คณะ					& วิศวกรรมศาสตร์ \\
	\bf สถานที่ติดต่อ				& 221 ถ.เพชรบุรีซอย 5 แขวงทุ่งพญาไท \\
							& เขตราชเทวี กรุงเทพ 10400 \\
	\bf โทรศัพท์				& 0-1377-3753 \\
	\bf อีเมล์					&  Nattaphon.Sa@student.chula.ac.th \\
							& \\
							& \\
	\bf อาจารย์ที่ปรึกษา			& อ.เชษฐ พัฒโนทัย \\
	\bf อาจารย์ที่ปรึกษาร่วม		& \\
							& \\
	\bf หน่วยงานที่ร่วมในโครงงาน	& \\
	\bf ตัวแทนหน่วยงาน			& \\
							& \\
	\bf คำสำคัญ (ภาษาไทย)		& \\
	\bf คำสำคัญ (ภาษาอังกฤษ)		& \\
\end{tabular}
}

\clearpage

\begin{center}
{\huge \bf ข้อเสนอโครงงานมหาบัณฑิต}
\end{center}

\noindent{{\LARGE \bf ชื่อหัวเรื่อง}}

%\Description{\MarginText{ภาษาไทยx}การพัฒนาระบบสำหรับการรวบรวมบัตรสมาชิก บนเทคโนโลยีเนียร์ฟิลด์คอมมูนิเคชันด้วยแอนดรอยด์แพลตฟอร์ม}

\Large{\noindent\hspace{0.7cm}\setlength{\tabcolsep}{15pt}
\begin{tabular}{l l}    
	ภาษาไทย 		& การพัฒนาระบบสำหรับการรวบรวมบัตรสมาชิก บนเทคโนโลยีเนียร์ฟิลด์ \\
				& คอมมูนิเคชันด้วยแอนดรอยด์แพลตฟอร์ม \\
	ภาษาอังกฤษ	& Developing a NFC Based Integrated Member Card System for \\
				& Mobile Devices Using the Android Platform \\    
\end{tabular}
}

% Here are some multilingual Unicode fonts: this is Arabic text: {\A السلام عليكم}, this is Hebrew: {\H שלום}, 
\section{ที่มาและความสำคัญของปัญหา}

เป็นที่ทราบกันในปัจจุบันนี้ว่า ร้านค้าส่วนใหญ่ต่างมุ่งไปที่ซีอาร์เอ็ม (Customer relationship management: CRM) หรือพัฒนาด้านการจัดการลูกค้าสัมพันธ์ โดยมุ่งเน้นนำเสนอสินค้าบริการที่สร้างความสุข ก่อให้เกิดความชื่นชอบในตัวสินค้า ใช้สินค้าอย่างสม่ำเสมอ บอกกันปากต่อปาก ก่อให้เกิดความภักดีในตราสินค้า และเกิดความผูกพันอย่างลึกซึ้งในตราสินค้า ธุรกิจหลากหลายรูปแบบไม่ว่าจะเป็น ภัตตาคาร ร้านอาหาร ห้างสรรพสินค้า สายการบิน โรงภาพยนตร์ และสถานบันเทิง ต่างมุ่งประเด็นใช้กลยุทธ์ต่าง ๆ เพื่อสร้างสัมพันธ์ที่ดีกับลูกค้า กลยุทธ์หนึ่งในนั้นคือ การทำบัตรสมาชิกเพื่อเพิ่มสิทธิประโยชน์หรือส่วนลดให้กับลูกค้า

ข้อดีของการสมัครบัตรสมาชิกคือ ช่วยให้เข้าใจความต้องการ และการตอบสนองของลูกค้าในสินค้าหรือบริการ สามารถสร้างผลกําไรในธุรกิจอย่างมีประสิทธิภาพ และสามารถดึงดูดลูกค้าให้กลับมาอีกครั้ง ด้วยเหตุนี้เองร้านค้าต่าง ๆ จึงมุ่งเน้นการทำบัตรสมาชิกเป็นจำนวนมาก แต่สิ่งที่ตามมากลับพบปัญหาว่ากลยุทธ์การทำตลาดดังกล่าวกลับไม่ได้ผลอย่างที่ควรจะเป็น อันเนื่องมาจากร้านค้าทุกร้านต่างก็ทำบัตรสมาชิกเป็นของตัวเอง จึงไม่เกิดความแตกต่างในข้อได้เปรียบหรือเสียเปรียบ อีกทั้งยังเป็นภาระของต้นทุนที่ทุกร้านจะต้องจ่ายไปกับการทำบัตรสมาชิก ปัญหาที่เกิดขึ้นไม่ได้ส่งผลกระทบต่อร้านค้าเพียงอย่างเดียว  ยังส่งผลกระ \newline ทบต่อลูกค้าด้วย ลูกค้าจะต้องพกพาบัตรสมาชิกของร้านค้าต่าง ๆ เป็นจำนวนมากซึ่งไม่อำนวยความสะดวกในการพกพา

จากปัญหาที่กล่าวมาข้างต้น ปัจจุบันเทคโนโลยีก้าวเข้ามามีส่วนในชีวิตประจำวันของเรามากขึ้น ในแต่ละปีที่ผ่านไปจะเห็นว่ามีคนที่ใช้อุปกรณ์พกพากันมากขึ้น ส่งผลให้จำนวนการเติบโตของอุปกรณ์พกพาที่สูงขึ้นมาก ยิ่งไปกว่านั้นอุปกรณ์พกพาต่าง ๆ ได้ผนวกเข้ากับเทคโนโลยีสื่อสารไร้สาย ซึ่งจะช่วยรองรับการสื่อสารระหว่างเครื่องมืออิเล็กทรอนิกส์ในระยะใกล้ ๆ ด้วยปัจจัยเหล่านี้เองถือเป็นโอกาสในการทรานส์ฟอร์มข้อมูลบัตรสมาชิกของแต่ละร้านค้าต่าง ๆ ลงบนอุปกรณ์พกพา

โครงงานมหาบัณฑิตนี้นำเสนอระบบต้นแบบ (Prototype) สำหรับการรวบรวมบัตรสมาชิกบนสมาร์ทโฟนที่ผนวกเข้ากับเทคโนโลยีเอ็นเอฟซี (Near Field Communication: NFC) ซึ่งโครงงานนี้จะมุ่งเน้นไปที่ความสามารถใช้งาน (Usability) และความสามารถเชิงฟังก์ชั่น (Functionality) ของสมาร์ทโฟนให้สามารถทำหน้าที่แทนบัตรสมาชิกของร้านค้าต่าง ๆ ได้ โครงงานนี้มีจุดประสงค์เพื่อออกแบบและพัฒนาระบบสำหรับการรวบรวมบัตรสมาชิกโดยใช้เทคโนโลยีเนียร์ฟิลด์คอมมูนิเคชันบนแอนดรอยด์แพลตฟอร์ม เพื่อแก้ปัญหาการจัดการบัตรสมาชิกของทางร้านค้า และลดการพกพาบัตรสมาชิกของลูกค้า โดยระบบต้นแบบดังกล่าวที่ถูกพัฒนาขึ้นสามารถนำไปใช้กับร้านค้า ร้านอาหาร ห้างสรรพสินค้าต่าง ๆ ได้

%----------------------------------------------------------------------------------------

\section{ทฤษฏีที่เกี่ยวข้อง}

การวิเคราะห์และออกแบบระบบต้นแบบสำหรับการรวบรวมบัตรสมาชิกบนสมาร์ทโฟนที่ผนวกเข้ากับเทคโนโลยีเอ็นเอฟซี ผู้ทำโครงงานได้ศึกษาเรื่องที่เกี่ยวข้องดังต่อไปนี้เพื่อประกอบการทําโครงงานมหาบัณฑิต แบ่งเป็น ระบบปฏิบัติการแอนดรอยด์ เอสคิวไลท์ (SQLite) และเอ็นเอฟซี ซึ่งสามารถจําแนกเป็นหลักการและทฤษฎีที่เกี่ยวข้อง ดังนี้

\subsection{Android}
แอนดรอยด์เป็นระบบปฏิบัติการที่มีพื้นฐานอยู่บนระบบปฏิบัติการลินุกซ์ ถูกออกแบบมาสำหรับ \newline อุปกรณ์ที่ใช้จอสัมผัส เช่นสมาร์ทโฟน และแท็บเล็ตคอมพิวเตอร์ ถูกคิดค้นและพัฒนาโดยบริษัทแอนดรอยด์ (Android, Inc.) ซึ่งต่อมา บริษัทกูเกิล (Google, Inc.) ได้ทำการซื้อต่อบริษัทในปี พ.ศ. 2548 แอนดรอยด์ถูกเปิดตัวเมื่อ ปี พ.ศ. 2550 พร้อมกับการก่อตั้งโอเพนแฮนด์เซตอัลไลแอนซ์ ซึ่งเป็นกลุ่มของบริษัทผลิตฮาร์ดแวร์ ซอฟต์แวร์ และการสื่อสารคมนาคม ที่ร่วมมือกันสร้างมาตรฐานเปิด สำหรับอุปกรณ์พกพา โดยสมาร์ทโฟนที่ใช้ระบบปฏิบัติการแอนดรอยด์เครื่องแรกของโลกคือ เอชทีซี ดรีม วางจำหน่ายเมื่อปี พ.ศ. 2551

\subsection{SQLite}
เอสคิวไลท์เป็นระบบจัดการฐานข้อมูลเชิงสัมพันธ์ (Relational Database Management System: RDMS) บรรจุอยู่ในโปรแกรมขนาดเล็ก พัฒนาด้วยภาษาซี เป็นระบบฐานข้อมูลที่สามารถทำงานได้โดยไม่จำเป็นต้องพึ่งพาเซิร์ฟเวอร์ ซึ่งแตกต่างกับระบบฐานข้อมูลอื่น ๆ เหมาะกับแอปพลิเคชันที่สามารถทำงานได้ด้วยตัวเอง (Standalone) สามารถนำไปประยุกต์ใช้งานได้หลากหลาย เช่น ดิกชินนารี เว็บบราวเซอร์ แคตาล็อคสินค้า โปรแกรมแบบสอบถาม การเก็บข้อมูลที่ต้องการส่งเป็นไฟล์ข้อมูลทางเมล์หรือมือถือ เป็นต้น

\subsection{NFC}
เทคโนโลยีเอ็นเอฟซีเป็นเทคโนโลยีสื่อสารไร้สายระยะไกล้ โดยใช้คลื่นวิทยุความถี่สูง รองรับการสื่อสารสองทางระหว่างเครื่องมืออิเล็กทรอนิกส์ในระยะประมาณ 1 - 4 ซม. (10 ซม. ในทางทฤษฎี) ที่ใช้ได้ดีกับโครงสร้างพื้นฐานแบบไร้สัมผัส เอ็นเอฟซีถูกพัฒนาขึ้นโดยบริษัท Sony และ NXP โดยใช้คลื่นความถี่ 13.56 MHz. รับส่งข้อมูลด้วยความเร็ว 424 Kbps บนพื้นฐานมาตรฐานไอเอสโอ/ไออีซี 18092 NFC IP-1 และไอเอสโอ/ไออีซี 14443 (Philips MIFARE and Sony’s FeliCa) โดยมาตรฐานดังกล่าวได้เสนอโหมดการทำงานทั้งสามแบบที่แตกต่างกันดังรูปที่ 1 

\begin{enumerate}
  \item Reader/Writer mode โหมดนี้อุปกรณ์เอ็นเอฟซีสามารถทำตัวเสมือนเป็นเครื่องอ่านเขียน Contactless Smart Card (หรือบางครั้งเรียกว่า Tag) โดยจะสามารถอ่านข้อมูลจาก Tag ที่ติดอยู่ใน Smartposter หรือจุดให้บริการข้อมูลได้ โดยโหมดดังกล่าวสอดคล้องกับมาตรฐานไอเอสโอ/ไออีซี 14443
  \item  NFC Card Emulation Mode โหมดนี้จะทำงานเสมือนเป็นบัตร Contactless ซึ่งนั่นหมายความว่าอุปกรณ์มือถือตามมาตรฐานเอ็นเอฟซีจะทำตัวเป็นบัตร Contactless Smart Card เพื่อใช้ในการทำธุรกรรมต่าง ๆ ได้
  \item Peer-to-Peer Mode โหมดนี้จะทำการแลกเปลี่ยนข้อมูลระหว่างอุปกรณ์เอ็นเอฟซีด้วยกันเช่นนามบัตร รูปถ่าย แฟ้มข้อมูลอื่น ๆ โดยโหมดดังกล่าวสอดคล้องกับมาตรฐานไอเอสโอ/ไออีซี 18092
\end{enumerate}

\begin{figure}[ht!]
\centering
\includegraphics[width=90mm]{NFC_Operating_modes_and_standards.png}
\caption{โหมดทำงานของเอ็นเอฟซีและมาตรฐาน}
\label{overflow}
\end{figure}

ปัจจุบันบริษัททั้งสองได้ร่วมมือกับบริษัทผู้ผลิตและพัฒนาโทรศัพท์เคลื่อนที่จัดตั้งเป็น NFC Forum เพื่อให้เกิดการใช้งานในรูปแบบต่าง ๆ มากขึ้น ในระยะเริ่มแรกมีบริษัทโทรศัพท์มือถือชั้นนำของโลกประกาศนำเทคโนโลยีนี้มาใช้กับโทรศัพท์มือถือแล้ว เช่น Nokia, Samsung, Motorola เป็นต้น

การประยุกต์ใช้งานส่วนใหญ่มักนำเทคโนโลยีเอ็นเอฟซีมาใช้กับการชำระเงินที่ต้องการความรวดเร็วและมีมูลค่าไม่สูงมาก ซึ่งจะทำให้โทรศัพท์เคลื่อนที่สามารถใช้เพื่อการชำระเงิน โดยวิธีการแตะบนเครื่องอ่านหรือเครื่องชำระเงิน เช่น การให้บริการในร้านอาหารจานด่วน ร้านขายสินค้า ระบบการซื้อขายตั๋ว และระบบการแลกเปลี่ยนข้อมูลแบบ peer-to-peer เช่น เพลง เกม และรูปภาพ การชำระเงินค่าโดยสารในระบบขนส่งมวลชน เป็นต้น การชำระเงินแบบไร้สัมผัสนี้ก่อให้เกิดการชำระเงินที่ง่ายและรวดเร็ว ลดการเข้าคิวชำระเงินในร้านค้า ห้างสรรพสินค้า และร้านสะดวกซื้อต่าง ๆ

%----------------------------------------------------------------------------------------

\section{งานวิจัยที่เกี่ยวข้อง}

การวิเคราะห์และออกแบบระบบต้นแบบสำหรับการรวบรวมบัตรสมาชิกบนสมาร์ทโฟนที่ผนวกเข้ากับเทคโนโลยีเอ็นเอฟซี ผู้ทำโครงงานได้ศึกษางานวิจัยที่เกี่ยวข้องดังต่อไปนี้เพื่อประกอบการทําโครงงานมหาบัณฑิต ดังนี้

\subsection{Shopping Application System With Near Field Communication (NFC) Based on Android}

ผลงานวิจัย ~\ref{pap:1}  ได้นำเสนอระบบต้นแบบสำหรับการชอปปิ้ง (shopping) ในห้างสรรพสินค้าบนเทคโนโลยีเนียร์ฟิลด์คอมมูนิเคชันด้วยแอนดรอยด์แพลตฟอร์ม ซึ่งผู้ใช้งานจะทำการชอปปิ้งโดยเลือกสินค้าที่ต้องการ และทำการแท็กกับสินค้าที่มีชิปประทับไปกับสินค้า โดยมีแอนดรอยด์สมาร์ทโฟนเป็นตัวอ่านข้อมูลและรายละเอียดของสินค้า ผู้ใช้งานสามารถทำการเพิ่ม ลบจำนวนของสินค้า หรือทำการลบสินค้าที่ไม่ต้องการได้ นอกจากนี้ผู้ใช้งานจะทำการยืนยันจับจ่ายสินค้าโดยยืนยันกับผู้ขายซึ่งระบบจะทำการบันทึกรายการสินค้าที่จับจ่าย

\subsection{IDA-Pay: an innovative micro-payment system based on NFC technology for Android mobile devices}

~\ref{pap:2}

%----------------------------------------------------------------------------------------

\section{แนวคิดและวิธีการดำเนินงาน}

โครงงานมหาบัณฑิตนี้ต้องการนําเสนอการออกแบบและวิเคราะห์ระบบสำหรับการรวบรวมบัตรสมาชิกบนเทคโนโลยีเอ็นเอฟซีด้วยแอนดรอยด์แพลตฟอร์ม ซึ่งมีรายละเอียดดังนี้

\subsection{การวิเคราะห์ความต้องการของระบบต้นแบบ}

\subsection{การออกแบบระบบต้นแบบ}

เพื่อการออกแบบหน้าที่พื้นฐานของการทํางานระบบต้นแบบเพื่อสนับสนุนการวางแผนโครงการ จึงใช้แนวความคิดเชิงวัตถุ (Object-Oriented Paradigm) ในการวิเคราะห์และออกแบบ โดยใช้แผนภาพ ยูสเคสและแผนภาพกิจกรรมแสดงหน้าที่การทํางานของระบบต้นแบบ

\subsubsection{ออกแบบฐานข้อมูลเชิงสัมพันธ์}

เพื่อแสดงความสัมพันธ์ของฐานข้อมูลและแบบจําลองข้อมูลที่ใช้สําหรับอธิบายถึงโครงสร้างและ ความสัมพันธ์ระหว่างข้อมูลภายในฐานข้อมูล ออกแบบ โดยใช้แผนภาพคลาสของระบบต้นแบบ

\subsubsection{ออกแบบส่วนต่อประสานกับผู้ใช้}

เพื่อแสดงโครงสร้าง และองค์ประกอบของหน้าจอการทํางานที่จะปรากฏในระบบต้นแบบ รวมถึงข้อความที่ใช้แสดงเตือนหรือแสดงความผิดพลาด และข้อความช่วยเหลือผู้ใช้งาน โดยแสดงเป็นภาพ ต้นแบบ (Prototype)

\subsection{พัฒนาระบบต้นแบบ}
\subsection{ทดสอบและตรวจสอบคุณภาพของระบบ}

การทดสอบระบบต้นแบบมีเป้าหมายเพื่อเพื่อตรวจสอบการทํางานที่ผิดพลาด และการทํางานไม่ถูกต้องของระบบภายใต้การทดสอบภายใน (Internal Testing) และภายนอก (External Testing) รวมถึงสภาพแวดล้อมสําหรับการทดสอบและกรณีที่ทดสอบ ซึ่งจะนํามาเป็นผลสรุปการทดสอบระบบต้นแบบ

\subsection{เรียบเรียงและจัดทําเอกสารของระบบ}

\section{วัตถุประสงค์}

โครงงานมหาบัณฑิตนี้มีวัตถุประสงค์เพื่อพัฒนาระบบสำหรับการรวบรวมบัตรสมาชิกบนเทคโนโลยีเนียร์ฟิลด์คอมมูนิเคชันด้วยแอนดรอยด์แพลตฟอร์ม

\section{ขอบเขตการดำเนินงาน}

\section{ขั้นตอนการดำเนินงาน}

\subsection{ศึกษาข้อมูลพื้นฐานที่เกี่ยวข้องกับหัวข้อของโครงงานมหาบัณฑิต}
\subsection{ศึกษาเทคโนโลยีที่จะนํามาใช้ในการวิเคราะห์ ออกแบบ และพัฒนาระบบ}
\subsection{วิเคราะห์และออกแบบระบบต้นแบบ}
\subsubsection{ออกแบบระบบโดยใช้แผนภาพยูเอ็มแอล}
\subsubsection{ออกแบบฐานข้อมูลเชิงสัมพันธ์}
\subsubsection{ออกแบบส่วนต่อประสานกับผู้ใช้}
\subsection{พัฒนาระบบต้นแบบ}
\subsection{ทดสอบและตรวจสอบคุณภาพของระบบ}
\subsection{เรียบเรียงและจัดทําเอกสารของระบบ}

\section{ประโยชน์โครงงานที่คาดว่าจะได้รับ}

\section{รายการอ้างอิง}

\label{pap:1}
\label{pap:2}
gsgt

\end{document}  