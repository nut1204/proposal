%----------------------------------------------------------------------------------------
%  Setting latex
%----------------------------------------------------------------------------------------
\documentclass[12pt,a4paper]{article}

% Page dimension
% \hoffset = 0.5in
% \voffset = 0.5in
% \topmargin = 0mm
% \textheight = 680pt
% \textwidth = 168mm
% \columnsep = 6mm

% See geometry.pdf to learn the layout options. There are lots.
% \usepackage{geometry}
\usepackage[top=1in,bottom=1in,right=1.25in,left=1.25in]{geometry}      

% ... or a4paper or a5paper or ... 
% \geometry{letterpaper}

% Activate for for rotated page geometry
% \geometry{landscape} 					     

% Activate to begin paragraphs with an empty line rather than an indent
% \usepackage[parfill]{parskip}

\usepackage{graphicx}
\usepackage{amssymb}
\usepackage{fontspec,xltxtra,xunicode}
\defaultfontfeatures{Mapping=tex-text}
%\setromanfont[Mapping=tex-text]{Hoefler Text}
%\setsansfont[Scale=MatchLowercase,Mapping=tex-text]{Gill Sans}
%\setmonofont[Scale=MatchLowercase]{Andale Mono}
\setmainfont[Scale=1.0]{TH SarabunPSK} 
\XeTeXlinebreaklocale 'th_TH'

% add a dot after the section number
\usepackage{titlesec}
% \titlelabel{\thetitle.\quad}
% font-size
\titleformat*{\section}{\Large\bfseries}
\titleformat*{\subsection}{\Large}
\titleformat*{\subsubsection}{\Large}
\titleformat*{\paragraph}{\large\bfseries}
\titleformat*{\subparagraph}{\large\bfseries}

%Gantt chart package
\usepackage{pgfgantt}

% Enumerated list with square brackets
\usepackage{enumitem}				

% Paragraph indent and break
\usepackage{indentfirst}				    

% It prevents placing floats before the section
\usepackage[section]{placeins}

\setlength{\parindent}{1.27cm}
\setlength{\parskip}{1ex plus 0.5ex minus 0.2ex}

% Change caption name of figures, table, abstract 
\renewcommand{\figurename}{รูปที่}
\renewcommand\tablename{ตารางที่}
\renewcommand{\refname}{เอกสารอ้างอิง}

\usepackage{amssymb}					% http://ctan.org/pkg/amssymb
\usepackage{pifont}						% http://ctan.org/pkg/pifont
\newcommand{\cmark}{\ding{51}}
\newcommand{\xmark}{\ding{55}}

\tolerance=1
\emergencystretch=\maxdimen
\hyphenpenalty=10000
\hbadness=10000

%----------------------------------------------------------------------------------------
%  Begin document
%----------------------------------------------------------------------------------------
\begin{document}

\begin{titlepage}

\begin{center}
{\huge \bf ข้อเสนอโครงงานมหาบัณฑิต} 
\end{center}

\begin{center}
{\huge \bf (MASTER PROJECT PROPOSAL)} 
\end{center}

\vspace{1cm}
{\Large\noindent\hspace{0.0cm}\setlength{\tabcolsep}{15pt}
\begin{tabular}{l l}    
	\bf ชื่อเรื่อง (ภาษาไทย) 		& ระบบสมาชิกที่ใช้เทคโนโลยีเอ็นเอฟซีสำหรับแอนดรอยด์ \\
								& สมาร์ตโฟน กรณีศึกษา: ระบบร้านอาหาร \\
	\bf ชื่อเรื่อง (ภาษาอังกฤษ)	& An NFC Based Membership System for Android \\
								& Smartphone, Case Study for Restaurant System \\ 
								& \\
								& \\
	\bf เสนอโดย					& นายณัฐพล แซ่ลิ้ม \\
	\bf เลขประจำตัวนิสิต			& 557 09752 21 \\
	\bf สาขาวิชา					& วิศวกรรมซอฟต์แวร์ \\
	\bf ภาควิชา					& วิศวกรรมคอมพิวเตอร์ \\
	\bf คณะ						& วิศวกรรมศาสตร์ \\
	\bf สถานที่ติดต่อ				& 221 ถ.เพชรบุรีซอย 5 แขวงทุ่งพญาไท \\
								& เขตราชเทวี กรุงเทพ 10400 \\
	\bf โทรศัพท์					& 0-1377-3753 \\
	\bf อีเมล์					&  Nattaphon.Sa@student.chula.ac.th \\
								& \\
								& \\
	\bf อาจารย์ที่ปรึกษา			& อ.เชษฐ พัฒโนทัย \\
	\bf อาจารย์ที่ปรึกษาร่วม		& - \\
								& \\
	\bf หน่วยงานที่ร่วมในโครงงาน	& - \\
	\bf ตัวแทนหน่วยงาน			& - \\
								& \\
	\bf คำสำคัญ (ภาษาไทย)		& เอ็นเอฟซี, ระบบสมาชิก, การส่งข้อมูลแบบเพียร์ทูเพียร์ \\
	\bf คำสำคัญ (ภาษาอังกฤษ)		& NFC, MemberShip System, Peer-to-Peer Mode \\
\end{tabular}}

\end{titlepage}

\clearpage

%----------------------------------------------------------------------------------------
%  Introduction
%----------------------------------------------------------------------------------------
\begin{center}
{\huge \bf ข้อเสนอโครงงานมหาบัณฑิต}
\end{center}

\noindent{{\LARGE \bf ชื่อหัวเรื่อง}} \\

\Large{\noindent\hspace{0.7cm}\setlength{\tabcolsep}{15pt}
\begin{tabular}{l l}    
	ภาษาไทย 		& ระบบสมาชิกที่ใช้เทคโนโลยีเอ็นเอฟซีสำหรับแอนดรอยด์  \\
					& สมาร์ตโฟน กรณีศึกษา: ระบบร้านอาหาร \\
	ภาษาอังกฤษ		& an NFC Based Membership System for Android \\
					& Smartphone, Case Study for Restaurant System \\    
\end{tabular}
}

\section{ที่มาและความสำคัญของปัญหา}

เป็นที่ทราบกันในปัจจุบันนี้ว่า การแข่งขันทางธุรกิจของร้านค้าต่าง ๆ มีอัตราการแข่งขันที่สูงขึ้นมาก ร้านค้าส่วนใหญ่ต่างมุ่งเน้นไปที่การพัฒนาด้านการจัดการลูกค้าสัมพันธ์ (Customer relationship management: CRM) วัตถุประสงค์หลักคือ มุ่งเน้นนำเสนอสินค้าบริการที่สร้างความสุข ก่อให้เกิดความชื่นชอบในตัวสินค้า ใช้สินค้าอย่างสม่ำเสมอ บอกกันปากต่อปาก ก่อให้เกิดความภักดีในตราสินค้า และเกิดความผูกพันอย่างลึกซึ้งในตราสินค้า ธุรกิจหลากหลายรูปแบบไม่ว่าจะเป็น ภัตตาคาร ร้านอาหาร ห้างสรรพสินค้า สายการบิน โรงภาพยนตร์ และสถานบันเทิง ต่างมุ่งประเด็นใช้กลยุทธ์ต่าง ๆ เพื่อสร้างสัมพันธ์ที่ดีกับลูกค้า กลยุทธ์หนึ่งในนั้นคือ การทำบัตรสมาชิกเพื่อเพิ่มสิทธิประโยชน์หรือส่วนลดให้กับลูกค้า

ข้อดีของการสมัครบัตรสมาชิกคือ ช่วยให้เข้าใจความต้องการ และการตอบสนองของลูกค้าในสินค้าหรือบริการ สามารถสร้างผลกําไรในธุรกิจอย่างมีประสิทธิภาพ และสามารถดึงดูดลูกค้าให้กลับมาอีกครั้ง ด้วยเหตุนี้เองร้านค้าต่าง ๆ จึงมุ่งเน้นการทำบัตรสมาชิกเป็นจำนวนมาก แต่สิ่งที่ตามมากลับพบปัญหาว่ากลยุทธ์การทำตลาดดังกล่าวกลับไม่ได้ผลอย่างที่คาดหวัง สืบเนื่องมาจากร้านค้าทุกร้านต่างก็ทำบัตรสมาชิกเป็นของตัวเอง ส่งผลให้ไม่เกิดความแตกต่างในข้อได้เปรียบหรือเสียเปรียบ อีกทั้งยังเป็นภาระของต้นทุนที่ทุกร้านจะต้องจ่ายไปกับการทำบัตรสมาชิก ปัญหาที่เกิดขึ้นไม่ได้ส่งผลเสียต่อร้านค้าเท่านั้น  ยังส่งผลกระทบต่อลูกค้าด้วย ลูกค้าจะต้องพกพาบัตรสมาชิกของร้านค้าต่าง ๆ เป็นจำนวนมากซึ่งไม่อำนวยความสะดวกในการพกพา

จากปัญหาที่กล่าวมาข้างต้น ปัจจุบันเทคโนโลยีก้าวเข้ามามีส่วนในชีวิตประจำวันของเรามากขึ้น ในแต่ละปีที่ผ่านไปจะเห็นว่ามีคนที่ใช้อุปกรณ์พกพากันมากขึ้น ส่งผลให้จำนวนการเติบโตของอุปกรณ์พกพาที่สูงขึ้นมาก \cite{itm:shopping} ยิ่งไปกว่านั้นอุปกรณ์พกพาต่าง ๆ ได้ผนวกเข้ากับเทคโนโลยีสื่อสารไร้สาย ซึ่งจะช่วยรองรับการสื่อสารของข้อมูลระหว่างเครื่องมืออิเล็กทรอนิกส์ในระยะใกล้ ๆ \cite{itm:rpp-mobile} ด้วยปัจจัยเหล่านี้เองถือเป็นโอกาสในการแปลงข้อมูลบัตรสมาชิกของแต่ละร้านค้าต่าง ๆ ลงบนอุปกรณ์พกพา 

โครงงานมหาบัณฑิตนี้นำเสนอต้นแบบ (Prototype) ของระบบสมาชิกโดยใช้เทคโนโลยีเอ็นเอฟซี (Near Field Communication: NFC) ที่ผนวกเข้ากับสมาร์ตโฟน โดยมุ่งเน้นที่ความสามารถในการใช้งาน (Usability) และความสามารถเชิงฟังก์ชันของระบบ (Functionality) เพื่อให้สมาร์ตโฟนสามารถทำหน้าที่แทนบัตรสมาชิกของร้านค้าต่าง ๆ ได้ โครงงานนี้มีวัตถุประสงค์เพื่อลดการพกพาบัตรสมาชิกของลูกค้า ช่วยแก้ปัญหาในกรณีที่บัตรสมาชิกเกิดสูญหาย และลดต้นทุนการผลิตบัตรสมาชิกของทางร้านค้า อีกทั้งยังช่วยในการส่งเสริมการขาย โดยระบบต้นแบบดังกล่าวที่ถูกพัฒนาขึ้นจะใช้ระบบสมาชิกของร้านอาหารเป็นกรณีศึกษา

%----------------------------------------------------------------------------------------
%  Overview on NFC Technology
%----------------------------------------------------------------------------------------
\section{ทฤษฏีที่เกี่ยวข้อง}
การวิเคราะห์และออกแบบระบบต้นแบบสำหรับการรวบรวมบัตรสมาชิกบนสมาร์ตโฟนที่ผนวกเข้ากับเทคโนโลยีเอ็นเอฟซี ผู้ทำโครงงานได้ศึกษาเรื่องที่เกี่ยวข้อง เพื่อประ\mbox{กอบ}การทําโครงงานมหาบัณฑิต แบ่งเป็น ระบบปฏิบัติการแอนดรอยด์ (Android) เอสคิวไลท์ (SQLite) เทคโนโลยีเอ็นเอฟซี โปรแกรมประยุกต์แสตมป์ (Stamp) โปรแกรมประยุกต์\mbox{กอต}อิต (Got it) และโปรแกรมประยุกต์แมคโดนัลด์ (mcdonald's) ซึ่ง\mbox{สามารถ}จําแนกเป็นหลักการและทฤษฎีที่เกี่ยวข้อง ดังนี้

\subsection{แอนดรอยด์}
แอนดรอยด์เป็นระบบปฏิบัติการที่มีพื้นฐานอยู่บนระบบปฏิบัติการลินุกซ์ ถูกออกแบบมาสำหรับอุปกรณ์ที่ใช้จอสัมผัส เช่นสมาร์ตโฟน และแท็บเล็ตคอมพิวเตอร์ ริเริ่มคิดค้นและพัฒนามาจากบริษัทแอนดรอยด์ (Android, Inc.) ซึ่งต่อมาบริษัทกูเกิล (Google, Inc.) ได้ทำการซื้อต่อบริษัทในปี พ.ศ. 2548 แอนดรอยด์ถูกเปิดตัวเมื่อ ปี พ.ศ. 2550 พร้อมกับการก่อตั้งโอเพนแฮนด์เซตอัลไลแอนซ์ (Open Handset Alliance) ซึ่งเป็นกลุ่มของบริษัทผลิตฮาร์ดแวร์ ซอฟต์แวร์ และการสื่อสารคมนาคม ที่ร่วมมือกันสร้างมาตรฐานเปิด สำหรับอุปกรณ์พกพา โดยสมาร์ตโฟนที่ใช้ระบบปฏิบัติการแอนดรอยด์เครื่องแรกของโลกคือ เอชทีซี ดรีม วางจำหน่ายเมื่อปี พ.ศ. 2551

\subsection{เอสคิวไลท์}
เอสคิวไลท์เป็นระบบจัดการฐานข้อมูลเชิงสัมพันธ์ (Relational Database Management System: RDMS) บรรจุอยู่ในโปรแกรมขนาดเล็ก พัฒนาด้วยภาษาซี เป็นระบบฐานข้อมูลที่สามารถทำงานได้โดยไม่จำเป็นต้องพึ่งพาเซิร์ฟเวอร์ ซึ่งแตกต่างกับระบบฐานข้อมูลอื่น ๆ เหมาะกับแอปพลิเคชันที่สามารถทำงานได้ด้วยตัวเอง (Standalone) สามารถนำไปประยุกต์ใช้งานได้หลากหลาย เช่น ดิกชินนารี เว็บบราวเซอร์ แคตาล็อคสินค้า โปรแกรมแบบสอบถาม การเก็บข้อมูลที่ต้องการส่งเป็นไฟล์ข้อมูลผ่านทางเมล์หรือสมาร์ตโฟน เป็นต้น

\subsection{เทคโนโลยีเอ็นเอฟซี}
เทคโนโลยีเอ็นเอฟซีเป็นเทคโนโลยีสื่อสารไร้สายระยะใกล้ โดยใช้คลื่นวิทยุความถี่สูง รองรับการสื่อสารสองทางระหว่างเครื่องมืออิเล็กทรอนิกส์ในระยะประมาณ 1 - 4 ซม. (10 ซม. ในทางทฤษฎี) ที่ใช้ได้ดีกับโครงสร้างพื้นฐานแบบไร้สัมผัส เอ็นเอฟซีถูกพัฒนาขึ้นโดยบริษัทโซนี (Sony, Inc.) และบริษัทเอ็นเอ็กพี (NXP, Inc.) โดยใช้คลื่นความถี่ 13.56 MHz. รับส่งข้อมูลด้วยความเร็ว 424 Kbps บนพื้นฐานมาตรฐานไอเอสโอ/ไออีซี 18092 NFC IP-1 \cite{itm:prp-rfid} และไอเอสโอ/ไออีซี 14443 \cite{itm:cicc} (Philips MIFARE and Sony’s FeliCa) โดยมาตรฐานดังกล่าวได้เสนอโหมดการทำงานทั้งสามแบบ \cite{itm:IDA-Pay} ที่แตกต่างกันดังรูปที่ \ref{fig:nfc}

\begin{enumerate}
\item อ่าน/เขียน (Reader/Writer mode) โหมดนี้อุปกรณ์เอ็นเอฟซีสามารถทำตัวเสมือนเป็นเครื่องอ่าน และเขียนบัตรสมาร์ทคาร์ดแบบไร้สัมผัส (Contactless Smart Card) หรือบางครั้งเรียกว่าแท็ก (Tag) โดยจะสามารถอ่านข้อมูลจากแท็กที่ติดอยู่ในจุดให้บริการข้อมูลได้ (Smartposter) ซึ่งโหมดดังกล่าวสอดคล้องกับมาตรฐานไอเอสโอ/ไออีซี 14443

\item เอ็นเอฟซีการ์ดอีมูเลชั่น (NFC Card Emulation Mode) โหมดนี้จะทำงานเสมือนเป็นบัตรสมาร์ทคาร์ดแบบไร้สัมผัส ซึ่งนั่นหมายความว่าอุปกรณ์ที่สอดคล้องตามมาตรฐานเอ็นเอฟซี สามารถทำตัวเป็นบัตรจะทำตัวเป็นบัตรสมาร์ทคาร์ดแบบไร้สัมผัส เพื่อใช้ในการทำธุรกรรมต่าง ๆ ได้ อย่างไรก็ตามแอนดรอยด์สมาร์ตโฟนไม่สามารถใช้โหมดนี้ได้ \cite{itm:IDA-Pay} เนื่องจากโหมดดังกล่าวถูกจำกัดให้ใช้ได้เฉพาะโปรแกรมประยุกต์กูเกิลวอลเล็ต (Google Wallet application) เท่านั้น 

\item เพียร์ทูเพียร์ (Peer-to-Peer Mode) โหมดนี้จะทำการแลกเปลี่ยนข้อมูลกันระหว่างอุปกรณ์เอ็นเอฟซี เช่น ข้อมูลนามบัตร ภาพถ่าย หรือแฟ้มข้อมูลอื่น ๆ ซึ่งโหมดดังกล่าวสอดคล้องตามมาตรฐานไอเอสโอ/ไออีซี 18092
\end{enumerate}

\begin{figure}[ht!]
\centering
\includegraphics[width=140mm]{nfc_operating_modes.png}
\caption{โหมดทำงานของเอ็นเอฟซีและมาตรฐาน} \label{fig:nfc}
\label{overflow}
\end{figure}

ปัจจุบันทั้งสองบริษัทได้ร่วมมือกับบริษัทผู้ผลิต และพัฒนาสมาร์ตโฟน จัดตั้งเอ็นเอฟซีฟอรัม (NFC Forum) เพื่อให้เทคโนโลยีเอ็นเอฟซีเกิดการใช้งานในหลากหลายรูปแบบมากยิ่งขึ้น ซึ่งในระยะเริ่มแรกบริษัทชั้นนำของโลกเช่น โนเกีย (Nokia) ซัมซุง (Samsung) และโมโตโรล่า (Motorola) ได้ริเริ่มนำเทคโนโลยีเอ็นเอฟซีมาผนวกเข้ากับสมาร์ตโฟนแล้ว

\subsection{โปรแกรมประยุกต์แสตมป์}
โปรแกรมประยุกต์แสตมป์ เป็นโปรแกรมที่ใช้สำหรับสะสมแต้ม เพื่อรับส่วนลดหรือสิทธิประโยชน์จากทางร้านค้าต่าง ๆ โดยโปรแกรมดังกล่าวทำงานร่วมกับฮาร์ดแวร์พิเศษ เมื่อลูกค้าซื้อสินค้าหรือบริการกับทางร้านค้าที่รองรับโปรแกรมประยุกต์แสตมป์ ร้านค้าจะใช้ฮาร์ดแวร์ที่พัฒนาขึ้น ประทับตราลงบนหน้าจอสมาร์ตโฟนของลูกค้า เพื่อใช้ในการยืนยันสิทธิ์หรือสะสมแต้ม รูปที่ \ref{fig:stamp} แสดงฮาร์ดแวร์ที่ใช้งานร่วมกับโปรแกรมประยุกต์แสตมป์ ซึ่งวัตถุประสงค์ของโปรแกรมนี้คือ เพื่อช่วยลดการพกพาบัตรสมาชิกลูกค้า ช่วยแก้ปัญหาในกรณีที่บัตรสมาชิกเกิดสูญหาย ทั้งยังช่วยในการส่งเสริมการขาย อย่างไรก็ตามฮาร์ดแวร์ดังกล่าวมีราคาค่อนข้างสูง
\begin{figure}[ht!]
\centering
\includegraphics[width=80mm]{stamp.png}
\caption{ฮาร์ดแวร์ที่ใช้งานร่วมกับโปรแกรมประยุกต์แสตมป์ \cite{itm:stamp}} \label{fig:stamp}
\label{overflow}
\end{figure}

\subsection{โปรแกรมประยุกต์กอตอิต}
โปรแกรมประยุกต์กอตอิต เป็นโปรแกรมที่ใช้สำหรับสะสมแต้ม เพื่อรับส่วนลดหรือสิทธิประโยชน์จากทางร้านค้าต่าง ๆ ที่ทำงานร่วมกับรหัสคิวอาร์ (QR Code) ซึ่งเป็นบาร์โค้ดสองมิติชนิดหนึ่ง ลูกค้าสามารถอ่านรหัสคิวอาร์ได้ด้วยเครื่องสแกนคิวอาร์ผ่านทางสมาร์ตโฟน ในการยืนยันสิทธิหรือสะสะแต้ม ดังรูปที่ \ref{fig:gotit} วัตถุประสงค์ของโปรแกรมคือ สำหรับลูกค้า ช่วยลดการพกพาบัตรสมาชิก สามารถหาโปรโมชั่นร้านค้า ณ ตำแหน่งที่เคียงได้ สำหรับร้านค้า ช่วยในการกระตุ้นยอดขายและเพิ่มฐานลูกค้า อย่างไรก็ตามโปรแกรมดังกล่าวไม่สามารถอ่านรหัสคิวอาร์ในสภาวะที่มีแสงน้อย หรือได้ไม่ดีเท่าที่ควร เนื่องจากการอ่านรหัสคิวอาร์ จะประมวลผลจากรูปภาพ และจำเป็นต้องอาศัยกล้องของสมาร์ตโฟนเพื่อใช้ในการถ่ายภาพ
\begin{figure}[ht!]
\centering
\includegraphics[width=140mm]{gotit.png}
\caption{โปรแกรมประยุกต์กอตอิต \cite{itm:gotit}} \label{fig:gotit}
\label{overflow}
\end{figure}

\subsection{โปรแกรมประยุกต์แมคโดนัลด์}
โปรแกรมประยุกต์แมคโดนัลด์ เป็นโปรแกรมสำหรับโปรโมทสินค้าและอาหารของร้านแมคโดนัลด์ รับส่วนลดหรือสิทธิประโยชน์ต่าง ๆ โดยโปรแกรมดังกล่าวทำงานร่วมกับรหัสคิวอาร์ ในการรับสิทธิประโยชน์จากร้านค้า และใช้การระบุตำแหน่งเพื่อค้นหาร้านที่อยู่ไกล้เคียงได้ รูปที่ \ref{fig:mcdonald} แสดงถึงโปรแกรมประยุกต์แมคโดนัลด์ 
\begin{figure}[ht!]
\centering
\includegraphics[width=140mm]{mcdonald.png}
\caption{โปรแกรมประยุกต์แมคโดนัลด์ \cite{itm:mcdonald}} \label{fig:mcdonald}
\label{overflow}
\end{figure}

%----------------------------------------------------------------------------------------
%  Related Works
%----------------------------------------------------------------------------------------
\section{งานวิจัยที่เกี่ยวข้อง}
การวิเคราะห์และออกแบบระบบต้นแบบสำหรับการรวบรวมบัตรสมาชิกบนสมาร์ตโฟนที่ผนวกเข้ากับเทคโนโลยีเอ็นเอฟซี ผู้ทำโครงงานได้ศึกษางานวิจัยที่เกี่ยวข้องเพื่อประ\mbox{กอบ}การทําโครงงานมหาบัณฑิต ดังนี้

งานวิจัยของ Husni และคณะ \cite{itm:shopping} ได้นำเสนอระบบต้นแบบสำหรับการชอปปิ้ง (shopping) ในห้างสรรพสินค้าบนเทคโนโลยีเอ็นเอฟซีด้วยแอนดรอยด์แพลตฟอร์ม ผู้ใช้งานสามารถทำการชอปปิ้งโดยเลือกสินค้าที่ต้องการ และทำการแท็กกับสินค้าที่มีชิปประทับไปกับสินค้า โดยมีแอนดรอยด์สมาร์ตโฟนเป็นตัวอ่านข้อมูลและรายละเอียดของสินค้า ผู้ใช้งานสามารถทำการเพิ่ม ลบจำนวนของสินค้า หรือทำการลบสินค้าที่ไม่ต้องการได้ นอกจากนี้ผู้ใช้งานสามารถทำการยืนยันสินค้ากับผู้ขายเพื่อชำระสินค้า ซึ่งระบบจะตรวจสอบการยืนยันตัวบุคคลด้วยรหัสลับบุคคล (Personal Identification Number: PIN) จากนั้นจึงส่งข้อมูลการชำระสินค้า โดยใช้ระบบการแลกเปลี่ยนข้อมูลแบบเพียร์ทูเพียร์ และทำการบันทึกรายการสินค้า งานวิจัยนี้เน้นการส่งข้อมูลการชำระสินค้าผ่านสมาร์ตโฟนด้วยเทคโนโลยีเอ็นเอฟซีซึ่งระบบสามารถทำงานได้แม้ในสภาวะออฟไลน์

งานวิจัยของ Mainetti และคณะ \cite{itm:IDA-Pay} ได้นำเสนอระบบต้นแบบสำหรับการชำระเงินที่มีความปลอดภัยบนเทคโนโลยีเอ็นเอฟซีดวยแอนดรอยด์แพลตฟอร์ม ผู้ใช้งานสามารถชำระเงินผ่านทางสมาร์ตโฟนได้ โดยระบบจะทำการส่งข้อมูลเลขที่บัตรเครดิต วันที่หมดอายุของบัตร รหัสซีวีวี (CVV) ไปยังอุปกรณ์ขายหน้าร้าน (POS) ข้อมูลดังกล่าวจะถูกเข้ารหัสแบบกุญแจสาธารณะ (Public key) พร้อมกับยอดเงินที่ต้องชำระ และส่งต่อไปยังเกตเวย์ (Gateway) โดยเกตเวย์เป็นเว็บเซิร์ฟเวอร์ ทำหน้าที่ถอดรหัส และส่งข้อมูลเกี่ยวกับการชำระเงินไปยังจุดให้บริการเครือข่ายบัตรเครดิตที่กำหนดไว้ (Credit Card Network Endpoint) เพื่อรับประกันความปลอดภัยของข้อมูล งานวิจัยนี้เน้นเรื่องการรักษาความปลอดภัยของข้อมูลการเงิน สร้างความมั่นใจให้กับผู้ใช้งาน

%----------------------------------------------------------------------------------------
%  System Architecture
%----------------------------------------------------------------------------------------
\section{แนวคิดและวิธีการดำเนินงาน}
โครงงานมหาบัณฑิตนี้นําเสนอการออกแบบและวิเคราะห์ระบบสำหรับการรวบรวมบัตรสมาชิกบนเทคโนโลยีเอ็นเอฟซีด้วยแอนดรอยด์แพลตฟอร์ม ซึ่งมีรายละเอียดดังนี้

\subsection{ศึกษาความต้องการของระบบ}
จากความต้องการของระบบสมาชิก ผู้ทำโครงการได้ศึกษาเพิ่มเติมจาก โปรแกรมประยุกต์แสตมป์ โปรแกรมประยุกต์กอตอิต และโปรแกรมประยุกต์แมคโดนัลด์ ศึกษาการใช้งานสมาร์ตโฟน การพัฒนาโปรแแกรมสมาร์ตโฟน รวมถึงศึกษาการพัฒนาโปรแกรมประยุกต์โดยใช้เทคโนโลยีเอ็นเอฟซี จึงได้มาเป็นรายการความต้องการของระบบต้นแบบสำหรับการรวบรวมบัตรสมาชิก ซึ่งมีฟังก์ชันการทำงานต่าง ๆ โดยมีรายละเอียดดังนี้
\subsubsection{ความต้องการเชิงหน้าที่ (Functional Requirements) ของระบบฝั่งลูกค้า}
\begin{itemize}
	\item ลูกค้าสามารถลงบันทึกเข้าสู่ระบบได้
	\item ลูกค้าสามารถสมัครสมาชิกกับร้านค้าได้
	\item ลูกค้าสามารถทำบัตรสมาชิกใหม่ได้
	\item ลูกค้าสามารถต่ออายุสมาชิกได้
	\item ลูกค้าสามารถใช้สิทธิ์สำหรับสมาชิกในการรับส่วนลด หรือสะสมแต้มได้
	\item ลูกค้าสามารถค้นหาร้านที่สนใจได้
	\item ลูกค้าสามารถเรียกดูโปรโมชั่น สิทธิ์พิเศษ หรือส่วนลดในร้านค้าที่สนใจได้
\end{itemize}

\subsubsection{ความต้องการเชิงหน้าที่ของระบบฝั่งร้านค้า}
\begin{itemize}
	\item ผู้ค้าสามารถลงบันทึกเข้าสู่ระบบได้
	\item ผู้ค้าสามารถตรวจสอบการเป็นสมาชิกได้
	\item ผู้ค้าสามารถกำหนดโปรโมชั่นได้
\end{itemize}

\subsection{แผนภาพยูสเคส (Use Case Diagram)}
จากรายการความต้องการของระบบสมาชิก แสดงด้วยแผนภาพยูสเคสดังรูปที่ \ref{fig:usecase}

\begin{figure}[ht!]
\centering
\includegraphics[width=140mm]{use_case.png}
\caption{แผนภาพยูสเคสของระบบสมาชิก} \label{fig:usecase}
\label{overflow}
\end{figure}

\subsection{วิเคราะห์และออกแบบระบบต้นแบบ}

จากการศึกษาและวิเคราะห์ความต้องการของระบบต้นแบบสำหรับการรวบรวมบัตรสมาชิก ผู้ทำโครงงานได้เสนอระบบ 2 แบบดังนี้

\subsubsection{ระบบสมาชิกแบบสแตนอโลน (Stand-alone)}
ระบบสมาชิกแบบสแตนอโลนเป็นระบบที่ใช้สมาร์ตโฟนทำตัวเสมือนเป็นเครื่องอ่าน โดยอ่านข้อมูลจากแท็กในร้านค้าเพื่อทำการยืนยันสมาชิก หรือสะสมแต้ม ข้อมูลของสมาชิกจะถูกเก็บไว้ในสมาร์ตโฟน ภาพที่ \ref{fig:rw_mode} แสดงภาพรวมของระบบสมาชิกแบบสแตนอโลน ข้อดีของระบบดังกล่าวคือ ระบบสามารถทำงานได้แม้ในสภาวะออฟไลน์ ระบบมีความซับซ้อนน้อยและง่ายต่อการพัฒนา แต่ข้อเสียคือ ระบบไม่มีโปรแกรมประยุกต์ในส่วนของร้านค้า ร้านค้าจะทราบสถานะการณ์การทำงานของระบบผ่านทางโปรแกรมของฝั่งลูกค้าเท่านั้น อีกทั้งระบบไม่มีการสำรองข้อมูล ในกรณีที่สมาร์ตโฟนเกิดสูญหาย ข้อมูลสมาชิกจะสูญหายตามไปด้วย

\begin{figure}[ht!]
\centering
\includegraphics[width=100mm]{rw_mode.png}
\caption{ภาพรวมของระบบสมาชิกแบบสแตนอโลน} \label{fig:rw_mode}
\label{overflow}
\end{figure}

\subsubsection{ระบบสมาชิกแบบไคลเอนต์เซิร์ฟเวอร์ (Client Server)}
ระบบสมาชิกแบบไคลเอนต์เซิร์ฟเวอร์ เป็นระบบที่ทำการแลกเปลี่ยนข้อมูลแบบเพียร์ทูเพียร์ระหว่างสมาร์ตโฟน เพื่อทำการยืนยันสมาชิกหรือสะสมแต้ม ข้อมูลสมาชิกจะถูกเก็บไว้ที่เซิร์ฟเวอร์ ภาพที่ \ref{fig:p2p_mode_cloud} แสดงภาพรวมของระบบสมาชิกแบบไคลเอนต์ เซิร์ฟเวอร์ ข้อดีของระบบดังกล่าวคือ ระบบมีโปรแกรมประยุกต์ทั้งฝั่งของลูกค้าและร้านค้า ร้านค้าสามารถทราบสถานะการณ์การทำงานของระบบผ่านทางโปรแกรมของฝั่งร้านค้าได้ และในกรณีที่สมาร์ตโฟนเกิดสูญหาย ข้อมูลสมาชิกจะยังคงอยู่ ลูกค้าสามารถนำสมาร์ตโฟนเครื่องใหม่มาทำการยืนยันเพื่อนำข้อมูลเดิมกลับมาได้ แต่ข้อเสียคือ ลูกค้าจะต้องทำการต่ออินเทอร์เน็ตเพื่อส่งข้อมูลสมาชิกไปทำการบันทึกข้อมูลที่เซิร์ฟเวอร์

% จากการวิเคราะห์ระบบสมาชิกที่ได้ออกแบบขึ้น ข้อได้เปรียบของระบบคือ ในขั้นตอนทำการสมัครสมาชิก และยืนยันสมาชิก ระบบสมาชิกดังกล่าวสามารถทำงานได้แม้ในสภาวะออฟไลน์ ทั้งลูกค้าและผู้ค้าไม่จำเป็นต้องค้นหาอินเทอร์เน็ตเพื่อใช้งาน และผู้ค้าสามารถมั่นใจได้ว่าข้อมูลสมาชิกที่ได้ มีความถูกต้อง น่าเชื่อถือ และไม่สามารถถูกปลอมแปลงได้ เพราะถ้าหากข้อมูลดังกล่าวสามารถถอดรหัสได้ นั้นแสดงว่าข้อมูลสมาชิกนั้นถูกเข้ารหัสจากร้านค้าที่เชื่อถือได้เท่านั้น อีกทั้งระบบยังสามารถทำงานได้ดีในสภาวะที่มีแสงน้อย ข้อเสียเปรียบของระบบนี้คือ ระบบดังกล่าวจะต้องทำงานในแอนดรอยด์สมาร์ตโฟนที่รองรับเทคโนโลยีเอ็นเอฟซีเท่านั้น

จากการวิเคราะห์ระบบสมาชิกที่ได้ออกแบบขึ้น ข้อได้เปรียบของระบบคือ ในขั้นตอนทำการสมัครสมาชิก และยืนยันสมาชิก ระบบสมาชิกดังกล่าวสามารถทำงานได้แม้ในสภาวะออฟไลน์ ทั้งลูกค้าและผู้ค้าไม่จำเป็นต้องค้นหาอินเทอร์เน็ตเพื่อใช้งาน อีกทั้งระบบยังสามารถทำงานได้ดีในสภาวะที่มีแสงน้อย ข้อเสียเปรียบของระบบนี้คือ ระบบดังกล่าวจะต้องทำงานในแอนดรอยด์สมาร์ตโฟนที่รองรับเทคโนโลยีเอ็นเอฟซีเท่านั้น


ข้อมูลสรุปการเปรียบเทียบคุณสมบัติของระบบระหว่างระบบสมาชิก โปรแกรมประยุกต์แสตมป์ โปรแกรมประยุกต์กอตอิต และโปรแกรมประยุกต์แมคโดนัลด์ แสดงดังในตารางที่ \ref{tab:compare_feature}

\begin{table}[ht!]
\centering
\resizebox{140mm}{!} {
\begin{tabular}{ | l | c | c | c | c |}
	\hline                        
	\bfseries{คุณสมบัติของระบบ / โปรแกรมประยุกต์} & \bfseries{ระบบสมาชิก} & \bfseries{แสตมป์} & \bfseries{กอตอิต} & \bfseries{แมคโดนัลด}์ \\
  	\hline 
  	มีโปรแกรมทั้งฝั่งลูกค้าและร้านค้า		& \cmark & \xmark & \xmark & \xmark \\
  	\hline
  	มีการสำรองข้อมูลของสมาชิก			& \cmark & \cmark & \cmark & \xmark \\
  	\hline
  	ทำงานได้ดีในสภาวะที่แสงน้อย 			& \cmark & \cmark & \xmark & \xmark \\
  	\hline
  	สามารถทำงานออฟไลน์ได้	 (ในขั้นตอนการสมัครและยืนยันสมาชิก) 	& \cmark & \cmark & \cmark & \xmark \\
  	\hline
  	ค้นหาร้านค้าที่ไกล้เคียงได้				& \xmark & \xmark & \cmark & \cmark \\
  	\hline
\end{tabular}
}
\caption{เปรียบเทียบคุณสมบัติของระบบระหว่างระบบสมาชิก โปรแกรมประยุกต์แสตมป์ โปรแกรมประยุกต์กอตอิต และโปรแกรมประยุกต์แมคโดนัลด์}
\label{tab:compare_feature}
\end{table}

\begin{figure}[ht!]
\centering
\includegraphics[width=120mm]{p2p_mode_cloud.png}
\caption{ภาพรวมของระบบสมาชิกแบบไคลเอนต์เซิร์ฟเวอร์} \label{fig:p2p_mode_cloud}
\label{overflow}
\end{figure}

\subsection{สถาปัตยกรรมของระบบ}
การออกแบบสถาปัตยกรรมของระบบ เพื่อให้ระบบต้นแบบที่พัฒนาขึ้นสามารถนำกลับมาใช้ซ้ำได้ (reusability) และง่ายต่อการบำรุงรักษา (maintainability) ผู้ทำโครงงานออกแบบโดยใช้สถาปัตยกรรมแบบ 3 เทียร์ (Three-tier architecture) ดังรูปที่ \ref{fig:architecture} โดยแบ่งการทำงานเป็น 3 ส่วน ได้แก่ ส่วนของข้อมูล (Data tier) ส่วนการประมวลผล (Application tier) และส่วนการแสดงผล (Presentation tier) ซึ่งมีรายละเอียดดังนี้

\begin{figure}[ht!]
\centering
\includegraphics[width=110mm]{architecture.png}
\caption{สถาปัตยกรรมของระบบสมาชิก} \label{fig:architecture}
\label{overflow}
\end{figure}

\begin{enumerate} 
\item ส่วนของข้อมูล ประกอบไปด้วยเครื่องบริการฐานข้อมูล (Server) ซึ่งทำหน้าที่ในการเก็บข้อมูลต่าง ๆ เช่น ข้อมูลสมาชิก ข้อมูลร้านค้า ซึ่งส่วนนี้จะเป็นอิสระกับส่วนการประมวลผล ช่วยเพิ่มความสามารถในการรองรับและต่อขยายระบบ (Scalability) และการปรับปรุงสมรรถนะของระบบได้ (Performance)

\item ส่วนการประมวลผล ประกอบไปด้วยโปรแกรม 2 ส่วน ซึ่งจะถูกติดตั้งในสมาร์ตโฟนที่รองรับเทคโนโลยีเอ็นเอฟซี โดยโปรแกรมของผั่งลูกค้าจะเก็บข้อมูลสมาชิกที่จำเป็นไว้ในระบบฐานข้อมูลเอสคิวไลท์ ใช้สำหรับการแลกเปลี่ยนข้อมูลกับโปรแกรมทางร้านค้าเพื่อทำการยืนยันข้อมูลสมาชิก
% \item ส่วนการประมวลผล ประกอบไปด้วยโปรแกรม 2 ส่วน ซึ่งจะถูกติดตั้งในสมาร์ตโฟนที่รองรับเทคโนโลยีเอ็นเอฟซี โดยโปรแกรมของผั่งลูกค้าจะเก็บข้อมูลสมาชิกที่จำเป็นไว้ในระบบฐานข้อมูลเอสคิวไลท์ ใช้สำหรับการแลกเปลี่ยนข้อมูลกับโปรแกรมทางร้านค้าเพื่อทำการยืนยันข้อมูลสมาชิก และโปรแกรมของฝั่งร้านค้าจะเก็บรหัสแบบกุญแจสาธารณะของร้านค้าต่าง ๆ เพื่อไว้ในใช้การตรวจสอบข้อมูลสมาชิก

\item ส่วนการแสดงผล ส่วนนี้เป็นส่วนบนสุดของโปรแกรม ทำหน้าที่ในการแสดงผลต่าง ๆ เช่นการแสดงหน้าจอในส่วนการยืนยันสมาชิก รายการร้านค้าที่ร่วมรายการ รายการข้อมูลที่ลูกค้าเป็นสมาชิกกับร้านค้า เป็นต้น
\end{enumerate}

\subsection{แผนภาพคลาส (Class diagram)}
แผนภาพคลาสแสดงถึงส่วนประกอบหลักของระบบ ซึ่งประกอบไปด้วย คลาสผู้ใช้ (User) คลาสลูกค้า (Customer) และคลาสผู้ค้า (Merchant) ทำหน้าที่เก็บข้อมูลฝั่งลูกค้าและผู้ค้า คลาสร้านค้า (Store) คลาสสมาชิก (Membership) และโปรโมชั่น (Promotion) ใช้สำหรับเก็บข้อมูลร้านค้า ข้อมูลสมาชิกและข้อมูลส่วนลดและโปรโมชั่นตามลำดับ ความสัมพันธ์ของแต่ละคลาสภายในแผนภาพคลาสแสดงดังรูปที่ \ref{fig:class_diagram}

\begin{figure}[ht!]
\centering
\includegraphics[width=100mm]{class_diagram.png}
\caption{แผนภาพคลาสของระบบสมาชิก} \label{fig:class_diagram}
\label{overflow}
\end{figure}

% เมื่อลูกค้าทำการสมัครสมาชิก ผู้ขายจะเรียกขอข้อมูลเลขบัตรประชาชน และหมายเลขโทรศัพท์จากลูกค้า จากนั้นผู้ขายจะกรอกข้อมูลดังกล่าวลงในโปรแกรมในส่วนของร้านค้า เมื่อใส่รายละเอียดเรียบร้อย และกดยืนยัน โปรแกรมจะทำการเข้ารหัสข้อมูลแบบกุญแจส่วนตัว (Private key) โดยอัลกอริทึมแบบ RSA ที่มีความยาว 2048 บิต 

% \begin{equation}M_{pri} = e_{pri}(M)\end{equation}

% \noindent ข้อมูลที่ถูกเข้ารหัสแบบกุญแจส่วนตัว จะทำให้ผู้ขายมั่นใจได้ว่าข้อมูลดังกล่าวถ้าหากถอดรหัสได้ถูกต้อง นั้นแสดงว่าข้อมูลดังกล่าวถูกเข้ารหัสจากร้านค้าเท่านั้น เมื่อข้อมูลเข้ารหัสเสร็จเรียบร้อย ข้อมูลจะถูกส่งไปยังโปรแกรมในส่วนของลูกค้าด้วยเทคโนโลยีเอ็นเอฟซีแบบเพียร์ทูเพียร์ ซึ่งจะถูกเก็บไว้ในระบบจัดการฐานข้อมูลเอสคิวไลท์ เพื่อใช้ในการยืนยันสมาชิกต่อไป รูปที่ \ref{fig:activity_regis} แสดงถึงแผนภาพกิจกรรมในขั้นตอนการสมัครสมาชิก

% \begin{figure}[ht!]
% \centering
% \includegraphics[width=140mm]{activity_regis.png}
% \caption{แผนภาพกิจกรรมในขั้นตอนการสมัครสมาชิก} \label{fig:activity_regis}
% \label{overflow}
% \end{figure}

% เมื่อลูกค้าสมัครสมาชิกแล้ว ครั้งต่อไปเมื่อลูกค้าเข้ามาใช้บริการ ลูกค้าจะต้องยืนยันสมาชิกเพื่อรับสิทธิพิเศษหรือสะสมแต้ม โดยลูกค้าจะเลือกร้านค้าที่เป็นสมาชิกจากรายการของร้านค้าที่เป็นสมาชิก จากนั้นโปรแกรมในส่วนของลูกค้าจะทำการดึงข้อมูลสมาชิกที่เข้ารหัสไว้ของร้านค้าดังกล่าว จากนั้นลูกค้าจะทำการส่งข้อมูลสมาชิกไปยังโปรแกรมในส่วนของร้านค้าเพื่อทำการถอดรหัส และตรวจสอบการเป็นสมาชิก รูปที่ \ref{fig:activity_regis} แสดงถึงแผนภาพกิจกรรมในขั้นตอนการยืนยันสมาชิก

% \begin{figure}[ht!]
% \centering
% \includegraphics[width=140mm]{activity_verify.png}
% \caption{แผนภาพกิจกรรมในขั้นตอนการยืนยันสมาชิก} \label{fig:activity_verify}
% \label{overflow}
% \end{figure}

\subsection{พัฒนาระบบต้นแบบ}
การพัฒนาระบบต้นแบบประกอบไปด้วย 3 ส่วนหลัก คือ โปรแกรมสำหรับลูกค้า โปรแกรมสำหรับร้านค้า และระบบเซิร์ฟเวอร์

\subsection{ทดสอบและตรวจสอบคุณภาพของระบบ}
การทดสอบระบบต้นแบบจะใช้การจำลองข้อมูลเพื่อทดสอบระบบเท่านั้น ข้อมูลที่จำลองประกอบไปด้วย ข้อมูลลูกค้า ข้อมูลผู้ค้า ข้อมูลโปรโมชั่น และข้อมูลร้านอาหารจำนวน 3 ร้าน การทดสอบจะมีการเพิ่มหรือปรับปรุงข้อมูลดังกล่าว เพื่อทดสอบว่าระบบสามารถทำงานได้

% การทดสอบระบบต้นแบบมีเป้าหมายเพื่อค้นหาข้อผิดพลาดที่มีอยู่ในโปรแกรม ตรวจสอบความถูกต้องของฟังก์ชันการทำงานของซอฟต์แวร์ (Verification) ตรวจสอบความถูกต้องของฟังก์ช้นการทํางานต่อความต้องการของผู้ใช้งาน (Validation) การทดสอบจะทำโดยการสร้างข้อมูลจำลองร้านค้าและข้อมูลของลูกค้า จากนั้นจะทำการส่งข้อมูลเพื่อทดสอบการสื่อสารระหว่างสมาร์ตโฟน ซึ่งจะนํามาเป็นผลสรุปการทดสอบระบบต้นแบบ

\subsection{เรียบเรียงและจัดทําเอกสารของระบบ}

%----------------------------------------------------------------------------------------
%  Proposed
%----------------------------------------------------------------------------------------
\section{วัตถุประสงค์}
โครงงานมหาบัณฑิตนี้มีวัตถุประสงค์เพื่อพัฒนาระบบสำหรับการรวบรวมบัตรสมาชิกบนเทคโนโลยีเนียร์ฟิลด์คอมมูนิเคชันด้วยแอนดรอยด์แพลตฟอร์ม ซึ่งช่วยลดการพกพาบัตรสมาชิกของลูกค้า แก้ปัญหาในกรณีที่บัตรสมาชิกเกิดสูญหาย และลดต้นทุนการผลิตบัตรสมาชิกของทางร้านค้า อีกทั้งยังช่วยในการส่งเสริมการขายอีกด้วย โดยระบบต้นแบบดังกล่าวที่ถูกพัฒนาขึ้นสามารถนำไปใช้กับร้านค้า ร้านอาหาร ห้างสรรพสินค้าต่าง ๆ ได้

\section{ขอบเขตการดำเนินงาน}
\subsection{ระบบไม่รองรับการค้นหาร้านค้า ณ ที่ตำแหน่งไกล้เคียง}
\subsection{การแลกเปลี่ยนข้อมูลระหว่างสมาร์ตโฟนด้วยเทคโนโลยีเอ็นเอฟซีจะใช้การส่งข้อมูลแบบเพียร์ทูเพียร์เท่านั้น}
\subsection{ระบบมีส่วนที่ติดต่อกับผู้ใช้งานในลักษณะที่เป็นกราฟิก (Graphic User Interface : GUI)}
\subsection{เครื่องมือที่ใช้ในการพัฒนาระบบ มีรายละเอียดดังต่อไปนี้}
\begin{enumerate}
	\item แมคบุ๊คแอร์ (Macbook Air) บนระบบปฏิบัติการแมคโอเอสเท็น (Max OS X) เวอร์ชั่น 10.8.4 ขึ้นไป ซึ่งติดตั้ง
	\begin{enumerate}
		\item Android Developer Tools (ADT) : adt--bundle--mac--x86\textunderscore64 --20130729.zip
		\item Eclipse Platform Version: 4.2.1
		\item Adobe Photoshop สำหรับออกแบบส่วนต่อประสานกับผู้ใช้งาน
	\end{enumerate}
  	\item สมาร์ตโฟนรุ่น LG Nexus 4 บนระบบปฏิบัติการแอนดรอยด์เวอร์ชั่น 4.3 ขึ้นไป ซึ่งผนวกเข้ากับเทคโนโลยีเอ็นเอฟซี
\end{enumerate}

\section{ขั้นตอนการดำเนินงาน}
\subsection{ศึกษาข้อมูลพื้นฐานที่เกี่ยวข้องกับหัวข้อของโครงงานมหาบัณฑิต}
\subsection{ศึกษาเทคโนโลยีที่จะนํามาใช้ในการวิเคราะห์ ออกแบบ และพัฒนาระบบ}
\subsection{วิเคราะห์และออกแบบระบบต้นแบบ}
\subsubsection{ออกแบบระบบโดยใช้แผนภาพยูเอ็มแอล}
\subsubsection{ออกแบบฐานข้อมูลเชิงสัมพันธ์}
\subsubsection{ออกแบบส่วนต่อประสานกับผู้ใช้}
\subsection{พัฒนาระบบต้นแบบ}
\subsection{ทดสอบและตรวจสอบคุณภาพของระบบ}
\subsection{เขียนบทความ}
\subsection{เรียบเรียงและจัดทําเอกสารของระบบ}

%\begin{figure}[ftbp]
\begin{center}
\begin{ganttchart}[y unit title=0.5cm,
y unit chart=0.5cm,
vgrid,hgrid, 
title label anchor/.style={below=-1.6ex},
title left shift=.05,
title right shift=-.05,
title height=1,
bar/.style={fill=gray!50},
bar label anchor/.append style={align=left, text width=6.5em},
incomplete/.style={fill=white},
progress label text={},
bar height=0.7,
group right shift=0,
group top shift=.6,
group height=.3,
group peaks={}{}{.2}]{18}
%labels
\gantttitle{ระยะเวลาในการดำเนินการ}{18} \\
\gantttitle{มิย. 56}{2}
\gantttitle{กค. 56}{2} 
\gantttitle{สค. 56}{2} 
\gantttitle{กย. 56}{2} 
\gantttitle{ตค. 56}{2} 
\gantttitle{พย. 56}{2}
\gantttitle{ธค. 56}{2}
\gantttitle{มค. 57}{2}
\gantttitle{กพ. 57}{2} \\
%tasks
\ganttbar{1. ศึกษาข้อมูลพื้นฐานที่เกี่ยวข้อง}{1}{2} \\
\ganttbar{2. ศึกษาเทคโนโลยีที่จะนํามาใช้}{3}{4} \\
\ganttbar{3. วิเคราะห์และออกแบบ}{5}{8} \\
\ganttbar{4. พัฒนาระบบต้นแบบ}{7}{11} \\
\ganttbar{5. ทดสอบและตรวจสอบ}{8}{13} \\
\ganttbar{6. เขียนบทความ}{12}{14} \\
\ganttbar{7. เรียบเรียงและจัดทําเอกสาร}{14}{18} \\
%\ganttbar[progress=33]{task 5}{20}{22} \\

%relations 
\ganttlink{elem0}{elem1}
\ganttlink{elem1}{elem2} 
\ganttlink{elem2}{elem3}
\ganttlink{elem3}{elem4}
\ganttlink{elem4}{elem5}
\ganttlink{elem5}{elem6}
\end{ganttchart}
\end{center}
%\caption{Gantt Chart}
%\end{figure}

\section{ประโยชน์โครงงานที่คาดว่าจะได้รับ}
ระบบต้นแบบที่พัฒนาขึ้นจะช่วยลดการพกพาบัตรสมาชิกของลูกค้า ช่วยแก้ปัญหาในกรณีที่บัตรสมาชิกเกิดสูญหาย และลดต้นทุนการผลิตบัตรสมาชิกของทางร้านค้า อีกทั้งยังช่วยในการส่งเสริมการขายอีกด้วย โดยระบบต้นแบบดังกล่าวที่ถูกพัฒนาขึ้นสามารถนำไปใช้กับร้านค้า ร้านอาหาร ห้างสรรพสินค้าต่าง ๆ ได้

\begin{thebibliography}{depth}

\bibitem{itm:shopping} Husni, E.; Purwantoro, S., "Shopping application system with Near Field Communication (NFC) based on Android," {\itshape System Engineering and Technology (ICSET), 2012 International Conference on} , vol., no., pp.1,6, 11-12 Sept. 2012
  
\bibitem{itm:IDA-Pay} Mainetti, L.; Patrono, L.; Vergallo, R., "IDA-Pay: An innovative micro-payment system based on NFC technology for Android mobile devices," {\itshape Software, Telecommunications and Computer Networks (SoftCOM), 2012 20th International Conference on} , vol., no., pp.1,6, 11-13 Sept. 2012

\bibitem{itm:rpp-mobile} Divyan M. Konidala, Made H. Dwijaksara, Kwangjo Kim, Dongman Lee, Daeyoung Kim, Byoungcheon Lee, and Soontae Kim, Resuscitating Privacy-Preserving Mobile Payment with Customer in Complete Control, {\itshape Journal of Personal and Ubiquitous Computing (PUC)}. Vol.16, pp.643-654, 2012
  
\bibitem{itm:prp-rfid} L. Catarinucci, S. Tedesco, D. De Donno, L. Tarricone: "Platform-Robust Passive UHF RFID Tags: a Case-Study in Robotics," {\itshape Progress In Electromagnetics Research C}, Vol. 30, 27-39, 2012.
  
\bibitem{itm:cicc} International Standard ISO/IEC 14443-1-2-3-4, Identification cards - Contactless integrated circuit cards - Proximity cards, 2008-07-15, ISO/IEC 2008, Switzerland.

\bibitem{itm:stamp} Mobiliti Co., Ltd., (2013, September 6). {\itshape Stamp Application}. Available: http://www.getmystamp.com/

\bibitem{itm:gotit} Got Apps Co., Ltd., (2013, September 6). {\itshape Gotit Application}. Available: https://www.you-got.it/

\bibitem{itm:mcdonald} McDonald's Restaurants Ltd., (2013, September 6). {\itshape Mcdonald Application}. Available: https://play.google.com/store/apps/details?id=com.md.mcfinder\&hl=en

\end{thebibliography}

\end{document}  