% XeLaTeX can use any Mac OS X font. See the setromanfont command below.
% Input to XeLaTeX is full Unicode, so Unicode characters can be typed directly into the source.

% The next lines tell TeXShop to typeset with xelatex, and to open and save the source with Unicode encoding.

%!TEX TS-program = xelatex
%!TEX encoding = UTF-8 Unicode

\documentclass[a4paper]{article}
\usepackage{geometry}				% See geometry.pdf to learn the layout options. There are lots.
\geometry{letterpaper}				% ... or a4paper or a5paper or ... 
%\geometry{landscape} 				% Activate for for rotated page geometry
%\usepackage[parfill]{parskip}    		% Activate to begin paragraphs with an empty line rather than an indent
\usepackage{graphicx}
\usepackage{amssymb}
\usepackage{fontspec,xltxtra,xunicode}
\defaultfontfeatures{Mapping=tex-text}
%\setromanfont[Mapping=tex-text]{Hoefler Text}
%\setsansfont[Scale=MatchLowercase,Mapping=tex-text]{Gill Sans}
%\setmonofont[Scale=MatchLowercase]{Andale Mono}
\setmainfont[Scale=1.1]{TH Sarabun New} 
\XeTeXlinebreaklocale 'th_TH'

\usepackage{titlesec}				% add a dot after the section number
\titlelabel{\thetitle.\quad}

\titleformat*{\section}{\Large\bfseries}
\titleformat*{\subsection}{\Large}
\titleformat*{\subsubsection}{\Large}
\titleformat*{\paragraph}{\large\bfseries}
\titleformat*{\subparagraph}{\large\bfseries}

\usepackage{pgfgantt}				%Gantt chart package
\usepackage{enumitem}				%Enumerated list with square brackets
\usepackage{indentfirst}				%Paragraph indent and break
\usepackage[section]{placeins}			%it prevents placing floats before the section
\setlength{\parindent}{1.27cm}
\setlength{\parskip}{1ex plus 0.5ex minus 0.2ex}

%page dimension
%\hoffset = 0.5in
%\voffset = 0.5in

\renewcommand{\figurename}{รูปที่}		%Change caption name of figures

% Will Robertson's fontspec.sty can be used to simplify font choices.
% To experiment, open /Applications/Font Book to examine the fonts provided on Mac OS X,
% and change "Hoefler Text" to any of these choices.

%\title{ข้อเสนอโครงงานมหาบัณฑิต (MASTER PROJECT PROPOSAL)}
%\author{The Author}
%\date{}							% Activate to display a given date or no date

% For many users, the previous commands will be enough.
% If you want to directly input Unicode, add an Input Menu or Keyboard to the menu bar 
% using the International Panel in System Preferences.
% Unicode must be typeset using a font containing the appropriate characters.
% Remove the comment signs below for examples.

% \newfontfamily{\A}{Geeza Pro}
% \newfontfamily{\H}[Scale=0.9]{Lucida Grande}
% \newfontfamily{\J}[Scale=0.85]{Osaka}
%----------------------------------------------------------------------------------------
\begin{document}
%\maketitle

\begin{center}
{\huge \bf ข้อเสนอโครงงานมหาบัณฑิต} 
\end{center}

\begin{center}
{\huge \bf (MASTER PROJECT PROPOSAL)} 
\end{center}

\vspace{1cm}
\Large{\noindent\hspace{0.2cm}\setlength{\tabcolsep}{15pt}
\begin{tabular}{l l}    
	\bf ชื่อเรื่อง (ภาษาไทย) 		& การพัฒนาระบบสำหรับการรวบรวมบัตรสมาชิกบนเทคโนโลยี \\
							& เนียร์ฟิลด์คอมมูนิเคชันด้วยแอนดรอยด์แพลตฟอร์ม \\
	\bf ชื่อเรื่อง (ภาษาอังกฤษ)		& Developing a NFC Based Integrated Member Card System \\
							& for Mobile Devices Using the Android Platform \\ 
							& \\
							& \\
	\bf เสนอโดย				& นายณัฐพล แซ่ลิ้ม \\
	\bf เลขประจำตัวนิสิต			& 557 09752 21 \\
	\bf สาขาวิชา				& วิศวกรรมซอฟต์แวร์ \\
	\bf ภาควิชา				& วิศวกรรมคอมพิวเตอร์ \\
	\bf คณะ					& วิศวกรรมศาสตร์ \\
	\bf สถานที่ติดต่อ				& 221 ถ.เพชรบุรีซอย 5 แขวงทุ่งพญาไท \\
							& เขตราชเทวี กรุงเทพ 10400 \\
	\bf โทรศัพท์				& 0-1377-3753 \\
	\bf อีเมล์					&  Nattaphon.Sa@student.chula.ac.th \\
							& \\
							& \\
	\bf อาจารย์ที่ปรึกษา			& อ.เชษฐ พัฒโนทัย \\
	\bf อาจารย์ที่ปรึกษาร่วม		& \\
							& \\
	\bf หน่วยงานที่ร่วมในโครงงาน	& - \\
	\bf ตัวแทนหน่วยงาน			& - \\
							& \\
	\bf คำสำคัญ (ภาษาไทย)		& เนียร์ฟิลด์คอมมูนิเคชัน \\
	\bf คำสำคัญ (ภาษาอังกฤษ)		& Near Field Communication \\
\end{tabular}
}

\clearpage

\begin{center}
{\huge \bf ข้อเสนอโครงงานมหาบัณฑิต}
\end{center}

\noindent{{\LARGE \bf ชื่อหัวเรื่อง}} \\

\Large{\noindent\hspace{0.7cm}\setlength{\tabcolsep}{15pt}
\begin{tabular}{l l}    
	ภาษาไทย 		& การพัฒนาระบบสำหรับการรวบรวมบัตรสมาชิก บนเทคโนโลยีเนียร์ฟิลด์ \\
				& คอมมูนิเคชันด้วยแอนดรอยด์แพลตฟอร์ม \\
	ภาษาอังกฤษ	& Developing a NFC Based Integrated Member Card System for \\
				& Mobile Devices Using the Android Platform \\    
\end{tabular}
}

% Here are some multilingual Unicode fonts: this is Arabic text: {\A السلام عليكم}, this is Hebrew: {\H שלום}, 
\section{ที่มาและความสำคัญของปัญหา}
เป็นที่ทราบกันในปัจจุบันนี้ว่า ร้านค้าส่วนใหญ่ต่างมุ่งไปที่ซีอาร์เอ็ม (Customer relationship management: CRM) หรือพัฒนาด้านการจัดการลูกค้าสัมพันธ์ โดยมุ่งเน้นนำเสนอสินค้าบริการที่สร้างความสุข ก่อให้เกิดความชื่นชอบในตัวสินค้า ใช้สินค้าอย่างสม่ำเสมอ บอกกันปากต่อปาก ก่อให้เกิดความภักดีในตราสินค้า และเกิดความผูกพันอย่างลึกซึ้งในตราสินค้า ธุรกิจหลากหลายรูปแบบไม่ว่าจะเป็น ภัตตาคาร ร้านอาหาร ห้างสรรพสินค้า สายการบิน โรงภาพยนตร์ และสถานบันเทิง ต่างมุ่งประเด็นใช้กลยุทธ์ต่าง ๆ เพื่อสร้างสัมพันธ์ที่ดีกับลูกค้า กลยุทธ์หนึ่งในนั้นคือ การทำบัตรสมาชิกเพื่อเพิ่มสิทธิประโยชน์หรือส่วนลดให้กับลูกค้า

ข้อดีของการสมัครบัตรสมาชิกคือ ช่วยให้เข้าใจความต้องการ และการตอบสนองของลูกค้าในสินค้าหรือบริการ สามารถสร้างผลกําไรในธุรกิจอย่างมีประสิทธิภาพ และสามารถดึงดูดลูกค้าให้กลับมาอีกครั้ง ด้วยเหตุนี้เองร้านค้าต่าง ๆ จึงมุ่งเน้นการทำบัตรสมาชิกเป็นจำนวนมาก แต่สิ่งที่ตามมากลับพบปัญหาว่ากลยุทธ์การทำตลาดดังกล่าวกลับไม่ได้ผลอย่างที่ควรจะเป็น อันเนื่องมาจากร้านค้าทุกร้านต่างก็ทำบัตรสมาชิกเป็นของตัวเอง จึงไม่เกิดความแตกต่างในข้อได้เปรียบหรือเสียเปรียบ อีกทั้งยังเป็นภาระของต้นทุนที่ทุกร้านจะต้องจ่ายไปกับการทำบัตรสมาชิก ปัญหาที่เกิดขึ้นไม่ได้ส่งผลกระทบต่อร้านค้าเพียงอย่างเดียว  ยังส่งผลกระ \newline ทบต่อลูกค้าด้วย ลูกค้าจะต้องพกพาบัตรสมาชิกของร้านค้าต่าง ๆ เป็นจำนวนมากซึ่งไม่อำนวยความสะดวกในการพกพา

จากปัญหาที่กล่าวมาข้างต้น ปัจจุบันเทคโนโลยีก้าวเข้ามามีส่วนในชีวิตประจำวันของเรามากขึ้น ในแต่ละปีที่ผ่านไปจะเห็นว่ามีคนที่ใช้อุปกรณ์พกพากันมากขึ้น ส่งผลให้จำนวนการเติบโตของอุปกรณ์พกพาที่สูงขึ้นมาก \ref{itm:shopping} ยิ่งไปกว่านั้นอุปกรณ์พกพาต่าง ๆ ได้ผนวกเข้ากับเทคโนโลยีสื่อสารไร้สาย ซึ่งจะช่วยรองรับการสื่อสารระหว่างเครื่องมืออิเล็กทรอนิกส์ในระยะใกล้ ๆ \ref{itm:rpp-mobile} ด้วยปัจจัยเหล่านี้เองถือเป็นโอกาสในการทรานส์ฟอร์มข้อมูลบัตรสมาชิกของแต่ละร้านค้าต่าง ๆ ลงบนอุปกรณ์พกพา 

โครงงานมหาบัณฑิตนี้นำเสนอระบบต้นแบบ (Prototype) สำหรับการรวบรวมบัตรสมาชิกบน \newline สมาร์ทโฟนที่ผนวกเข้ากับเทคโนโลยีเอ็นเอฟซี (Near Field Communication: NFC) ซึ่งโครงงานนี้จะมุ่งเน้นไปที่ความสามารถใช้งาน (Usability) และความสามารถเชิงฟังก์ชัน (Functionality) ของสมาร์ทโฟนให้สามารถทำหน้าที่แทนบัตรสมาชิกของร้านค้าต่าง ๆ ได้ โครงงานนี้มีจุดประสงค์เพื่อออกแบบและพัฒนาระบบดังกล่าว เพื่อแก้ปัญหาลดต้นทุนการผลิตบัตรสมาชิกของทางร้านค้า ลดการพกพาบัตรสมาชิกของลูกค้า และช่วยแก้ปัญหาในกรณีที่บัตรสมาชิกของลูกค้าเกิดศูนย์หาย โดยระบบต้นแบบดังกล่าวที่ถูกพัฒนาขึ้นสามารถนำไปใช้กับร้านค้า ร้านอาหาร ห้างสรรพสินค้าต่าง ๆ ได้

%----------------------------------------------------------------------------------------

\section{ทฤษฏีที่เกี่ยวข้อง}
การวิเคราะห์และออกแบบระบบต้นแบบสำหรับการรวบรวมบัตรสมาชิกบนสมาร์ทโฟนที่ผนวกเข้ากับเทคโนโลยีเอ็นเอฟซี ผู้ทำโครงงานได้ศึกษาเรื่องที่เกี่ยวข้องเพื่อประกอบการทําโครงงานมหาบัณฑิต แบ่งเป็น ระบบปฏิบัติการแอนดรอยด์ เอสคิวไลท์ (SQLite) และเอ็นเอฟซี ซึ่งสามารถจําแนกเป็นหลักการและทฤษฎีที่เกี่ยวข้อง ดังนี้

\subsection{Android}
แอนดรอยด์เป็นระบบปฏิบัติการที่มีพื้นฐานอยู่บนระบบปฏิบัติการลินุกซ์ ถูกออกแบบมาสำหรับ \newline อุปกรณ์ที่ใช้จอสัมผัส เช่นสมาร์ทโฟน และแท็บเล็ตคอมพิวเตอร์ ถูกคิดค้นและพัฒนาโดยบริษัทแอนดรอยด์ (Android, Inc.) ซึ่งต่อมา บริษัทกูเกิล (Google, Inc.) ได้ทำการซื้อต่อบริษัทในปี พ.ศ. 2548 แอนดรอยด์ถูกเปิดตัวเมื่อ ปี พ.ศ. 2550 พร้อมกับการก่อตั้งโอเพนแฮนด์เซตอัลไลแอนซ์ ซึ่งเป็นกลุ่มของบริษัทผลิตฮาร์ดแวร์ ซอฟต์แวร์ และการสื่อสารคมนาคม ที่ร่วมมือกันสร้างมาตรฐานเปิด สำหรับอุปกรณ์พกพา โดยสมาร์ทโฟนที่ใช้ระบบปฏิบัติการแอนดรอยด์เครื่องแรกของโลกคือ เอชทีซี ดรีม วางจำหน่ายเมื่อปี พ.ศ. 2551

\subsection{SQLite}
เอสคิวไลท์เป็นระบบจัดการฐานข้อมูลเชิงสัมพันธ์ (Relational Database Management System: RDMS) บรรจุอยู่ในโปรแกรมขนาดเล็ก พัฒนาด้วยภาษาซี เป็นระบบฐานข้อมูลที่สามารถทำงานได้โดยไม่จำเป็นต้องพึ่งพาเซิร์ฟเวอร์ ซึ่งแตกต่างกับระบบฐานข้อมูลอื่น ๆ เหมาะกับแอปพลิเคชันที่สามารถทำงานได้ด้วยตัวเอง (Standalone) สามารถนำไปประยุกต์ใช้งานได้หลากหลาย เช่น ดิกชินนารี เว็บบราวเซอร์ แคตาล็อคสินค้า โปรแกรมแบบสอบถาม การเก็บข้อมูลที่ต้องการส่งเป็นไฟล์ข้อมูลผ่านทางเมล์หรือสมาร์ทโฟน เป็นต้น

\subsection{NFC}
เทคโนโลยีเอ็นเอฟซีเป็นเทคโนโลยีสื่อสารไร้สายระยะไกล้ โดยใช้คลื่นวิทยุความถี่สูง รองรับการสื่อสารสองทางระหว่างเครื่องมืออิเล็กทรอนิกส์ในระยะประมาณ 1 - 4 ซม. (10 ซม. ในทางทฤษฎี) ที่ใช้ได้ดีกับโครงสร้างพื้นฐานแบบไร้สัมผัส เอ็นเอฟซีถูกพัฒนาขึ้นโดยบริษัท Sony และ NXP โดยใช้คลื่นความถี่ 13.56 MHz. รับส่งข้อมูลด้วยความเร็ว 424 Kbps บนพื้นฐานมาตรฐานไอเอสโอ/ไออีซี 18092 NFC IP-1 \ref{itm:prp-rfid} และไอเอสโอ/ไออีซี 14443 \ref{itm:cicc} (Philips MIFARE and Sony’s FeliCa) โดยมาตรฐานดังกล่าวได้เสนอรูปแบบการทำงานทั้งสามแบบ \ref{itm:IDA-Pay} ที่แตกต่างกันดังรูปที่ 1 

\begin{enumerate}
	\item อ่าน/เขียน (Reader/Writer mode) โหมดนี้อุปกรณ์เอ็นเอฟซีสามารถทำตัวเสมือนเป็นเครื่องอ่านเขียน Contactless Smart Card หรือบางครั้งเรียกว่าแท็ก (Tag) โดยจะสามารถอ่านข้อมูลจากแท็กที่ติดอยู่ใน Smartposter หรือจุดให้บริการข้อมูลได้ ซึ่งโหมดดังกล่าวสอดคล้องกับมาตรฐานไอเอสโอ/ไออีซี 14443
  	\item เอ็นเอฟซีการ์ดอีมูเลชั่น (NFC Card Emulation Mode) โหมดนี้จะทำงานเสมือนเป็นบัตร Contactless ซึ่งนั่นหมายความว่าอุปกรณ์สมาร์ทโฟนตามมาตรฐานเอ็นเอฟซีจะทำตัวเป็นบัตร Contactless Smart Card เพื่อใช้ในการทำธุรกรรมต่าง ๆ ได้
  	\item เพียร์ทูเพียร์ (Peer-to-Peer Mode) โหมดนี้จะทำการแลกเปลี่ยนข้อมูลระหว่างอุปกรณ์เอ็นเอฟซีด้วยกันเช่นนามบัตร รูปถ่าย แฟ้มข้อมูลอื่น ๆ ซึ่งโหมดดังกล่าวสอดคล้องกับมาตรฐานไอเอสโอ/ไออีซี 18092
\end{enumerate}

\begin{figure}[ht!]
\centering
\includegraphics[width=90mm]{NFC_Operating_modes_and_standards.png}
\caption{โหมดทำงานของเอ็นเอฟซีและมาตรฐาน}
\label{overflow}
\end{figure}

ปัจจุบันบริษัททั้งสองได้ร่วมมือกับบริษัทผู้ผลิตและพัฒนาสมาร์ทโฟนจัดตั้งเป็น NFC Forum เพื่อให้เกิดการใช้งานในรูปแบบต่าง ๆ มากขึ้น ในระยะเริ่มแรกมีบริษัทชั้นนำของโลกประกาศนำเทคโนโลยีนี้มาใช้กับสมาร์ทโฟนแล้ว เช่น Nokia, Samsung, Motorola เป็นต้น

การประยุกต์ใช้งานส่วนใหญ่มักนำเทคโนโลยีเอ็นเอฟซีมาใช้กับการชำระเงินที่ต้องการความรวดเร็วและมีมูลค่าไม่สูงมาก ซึ่งจะทำให้สมาร์ทโฟนสามารถใช้เพื่อการชำระเงิน โดยวิธีการแตะบนเครื่องอ่านหรือเครื่องชำระเงิน เช่น การให้บริการในร้านอาหารจานด่วน ร้านขายสินค้า ระบบการซื้อขายตั๋ว และระบบการแลกเปลี่ยนข้อมูลแบบเพียร์ทูเพียร์เช่น เพลง เกม และรูปภาพ การชำระเงินค่าโดยสารในระบบขนส่งมวลชน เป็นต้น การชำระเงินแบบไร้สัมผัสนี้ก่อให้เกิดการชำระเงินที่ง่ายและรวดเร็ว ลดการเข้าคิวชำระเงินในร้านค้า ห้างสรรพสินค้า และร้านสะดวกซื้อต่าง ๆ

%----------------------------------------------------------------------------------------

\section{งานวิจัยที่เกี่ยวข้อง}
การวิเคราะห์และออกแบบระบบต้นแบบสำหรับการรวบรวมบัตรสมาชิกบนสมาร์ทโฟนที่ผนวกเข้ากับเทคโนโลยีเอ็นเอฟซี ผู้ทำโครงงานได้ศึกษางานวิจัยที่เกี่ยวข้องเพื่อประกอบการทําโครงงานมหาบัณฑิต ดังนี้

\subsection{Shopping Application System With Near Field Communication (NFC) Based on Android}
ผลงานวิจัย \ref{itm:shopping}  ได้นำเสนอระบบต้นแบบสำหรับการชอปปิ้ง (shopping) ในห้างสรรพสินค้าบนเทคโนโลยีเนียร์ฟิลด์คอมมูนิเคชันด้วยแอนดรอยด์แพลตฟอร์ม ผู้ใช้งานสามารถทำการชอปปิ้งโดยเลือกสินค้าที่ต้องการ และทำการแท็กกับสินค้าที่มีชิปประทับไปกับสินค้า โดยมีแอนดรอยด์สมาร์ทโฟนเป็นตัวอ่านข้อมูลและรายละเอียดของสินค้า ผู้ใช้งานสามารถทำการเพิ่ม ลบจำนวนของสินค้า หรือทำการลบสินค้าที่ไม่ต้องการได้ นอกจากนี้ผู้ใช้งานสามารถทำการยืนยันจับจ่ายสินค้าโดยยืนยันกับผู้ขายซึ่งระบบจะตรวจสอบรหัสสำหรับการยืนยันตัวบุคคลด้วยรหัสลับบุคคล (Personal Identification Number: PIN) โดยใช้ระบบการแลกเปลี่ยนข้อมูลแบบเพียร์ทูเพียร์ และทำการบันทึกรายการสินค้าที่จับจ่าย

\subsection{IDA-Pay: an innovative micro-payment system based on NFC technology for Android mobile devices}
ผลงานวิจัย \ref{itm:IDA-Pay} ได้นำเสนอระบบต้นแบบสำหรับการชำระเงินที่มีความปลอดภัยบนเทคโนโลยีเนียร์ฟิลด์คอมมูนิเคชันด้วยแอนดรอยด์แพลตฟอร์ม โดยใช้ระบบการแลกเปลี่ยนข้อมูลแบบเพียร์ทูเพียร์ ซึ่งข้อมูลที่ถูกส่งไปยัง ณ จุดขาย ข้อมูลจะถูกเข้ารหัสแบบกุญแจสาธารณะ (Public key) และถูกส่งต่อไปยังเกตเวย์ (Gateway) ซึ่งเป็นเว็บเซิร์ฟเวอร์ที่จะส่งข้อมูลเกี่ยวกับการชำระเงินไปยังจุดให้บริการเครือข่ายบัตรเครดิตที่กำหนดไว้ (Credit Card Network Endpoint) เพื่อรับประกันความปลอดภัยของข้อมูล

%----------------------------------------------------------------------------------------

\section{แนวคิดและวิธีการดำเนินงาน}
โครงงานมหาบัณฑิตนี้นําเสนอการออกแบบและวิเคราะห์ระบบสำหรับการรวบรวมบัตรสมาชิกบนเทคโนโลยีเอ็นเอฟซีด้วยแอนดรอยด์แพลตฟอร์ม ซึ่งมีรายละเอียดดังนี้

\subsection{โครงสร้างสถาปัตยกรรม}
ระบบต้นแบบสำหรับการรวบรวมบัตรสมาชิก
\begin{figure}[ht!]
\centering
\includegraphics[width=150mm]{system_architecture.png}
\caption{โครงสร้างสถาปัตยกรรมของระบบต้นแบบ}
\label{overflow}
\end{figure}

\subsection{วิเคราะห์และออกแบบระบบต้นแบบ}
ในการวิเคราะห์และออกแบบระบบต้นแบบ เพื่อแสดงฟังก์ชันการทำงานของระบบ จึงใช้แนวความคิดเชิงวัตถุ (Object-Oriented Paradigm) ในการวิเคราะห์และออกแบบระบบต้นแบบ โดยใช้แผนภาพยูสเคส และแผนภาพกิจกรรมเพื่ออธิบายฟังก์ชันการทํางานของระบบ ในส่วนของการออกแบบฐานข้อมูลเชิงสัมพันธ์จะออกแบบโดยใช้แผนภาพคลาส เพื่อแสดงความสัมพันธ์ของฐานข้อมูล และแบบจําลองข้อมูลที่ใช้สําหรับอธิบายถึงโครงสร้าง และความสัมพันธ์ระหว่างข้อมูลภายในฐานข้อมูล และในส่วนของการออกแบบส่วนต่อประสานกับผู้ใช้งาน เพื่อแสดงถึงโครงสร้าง และองค์ประกอบของหน้าจอการทํางานที่จะปรากฏในระบบต้นแบบ รวมถึงข้อความที่ใช้แสดงเตือนหรือแสดงข้อผิดพลาด และข้อความช่วยเหลือผู้ใช้งาน โดยแสดงเป็นภาพต้นแบบ

\subsection{พัฒนาระบบต้นแบบ}
การพัฒนาระบบต้นแบบประกอบไปด้วย 2 ส่วนหลัก คือ โปรแกรมสำหรับลูกค้า และโปรแกรมสำหรับร้านค้า

\subsection{ทดสอบและตรวจสอบคุณภาพของระบบ}
การทดสอบระบบต้นแบบมีเป้าหมายเพื่อค้นหาข้อผิดพลาดที่มีอยู่ในโปรแกรม ตรวจสอบความถูกต้องของฟังก์ชันการทำงานของซอฟต์แวร์ (Verification) ตรวจสอบความถูกต้องของฟังก์ช้นการทํางานต่อความต้องการของผู้ใช้งาน (Validation) การทดสอบจะทำโดยการสร้างข้อมูลจำลองร้านค้าและข้อมูลของลูกค้า จากนั้นจะทำการส่งข้อมูลเพื่อทดสอบการสื่อสารระหว่างสมาร์ทโฟน ซึ่งจะนํามาเป็นผลสรุปการทดสอบระบบต้นแบบ

\subsection{เรียบเรียงและจัดทําเอกสารของระบบ}

%----------------------------------------------------------------------------------------

\section{วัตถุประสงค์}
โครงงานมหาบัณฑิตนี้มีวัตถุประสงค์เพื่อพัฒนาระบบสำหรับการรวบรวมบัตรสมาชิกบนเทคโนโลยีเนียร์ฟิลด์คอมมูนิเคชันด้วยแอนดรอยด์แพลตฟอร์ม

\section{ขอบเขตการดำเนินงาน}
\subsection{วิเคราะห์และออกแบบระบบต้นแบบสำหรับการรวบรวมบัตรสมาชิกบนสมาร์ทโฟนที่ผนวกเข้ากับเทคโนโลยีเอ็นเอฟซี ซึ่งมีฟังก์ชันการทำงานที่ประกอบไปด้วย}
\begin{itemize}
	\item The first item
	\item The second item
	\item The third etc \ldots
\end{itemize}
\subsection{ระบบมีการแลกเปลี่ยนข้อมูลระหว่างสมาร์ทโฟนด้วยเทคโนโลยีเอ็นเอฟซีโดยแบบเพียร์ทูเพียร์}
\subsection{ระบบมีส่วนที่ติดต่อกับผู้ใช้งานในลักษณะที่เป็นกราฟิก (Graphic User Interface : GUI)}
\subsection{เครื่องมือที่ใช้ในการพัฒนาระบบ มีรายละเอียดดังต่อไปนี้}
\begin{enumerate}
	\item แมคบุ๊คแอร์ (Macbook Air) บนระบบปฏิบัติการแมคโอเอสเท็น (Max OS X) เวอร์ชั่น 10.8.4 ขึ้นไป ซึ่งติดตั้ง
	\begin{enumerate}
		\item Android Developer Tools (ADT) : adt--bundle--mac--x86\textunderscore64--20130729.zip
		\item Eclipse Platform Version: 4.2.1
		\item Adobe Photoshop สำหรับออกแบบส่วนต่อประสานกับผู้ใช้งาน
	\end{enumerate}
  	\item สมาร์ทโฟนรุ่น LG Nexus 4 บนระบบปฏิบัติการแอนดรอยด์เวอร์ชั่น 4.3 ขึ้นไป ซึ่งผนวกเข้ากับเทคโนโลยีเอ็นเอฟซี
\end{enumerate}

\section{ขั้นตอนการดำเนินงาน}
\subsection{ศึกษาข้อมูลพื้นฐานที่เกี่ยวข้องกับหัวข้อของโครงงานมหาบัณฑิต}
\subsection{ศึกษาเทคโนโลยีที่จะนํามาใช้ในการวิเคราะห์ ออกแบบ และพัฒนาระบบ}
\subsection{วิเคราะห์และออกแบบระบบต้นแบบ}
\subsubsection{ออกแบบระบบโดยใช้แผนภาพยูเอ็มแอล}
\subsubsection{ออกแบบฐานข้อมูลเชิงสัมพันธ์}
\subsubsection{ออกแบบส่วนต่อประสานกับผู้ใช้}
\subsection{พัฒนาระบบต้นแบบ}
\subsection{ทดสอบและตรวจสอบคุณภาพของระบบ}
\subsection{เรียบเรียงและจัดทําเอกสารของระบบ}

%\begin{figure}[ftbp]
\begin{center}
\begin{ganttchart}[y unit title=0.5cm,
y unit chart=0.5cm,
vgrid,hgrid, 
title label anchor/.style={below=-1.6ex},
title left shift=.05,
title right shift=-.05,
title height=1,
bar/.style={fill=gray!50},
bar label anchor/.append style={align=left, text width=6.5em},
incomplete/.style={fill=white},
progress label text={},
bar height=0.7,
group right shift=0,
group top shift=.6,
group height=.3,
group peaks={}{}{.2}]{18}
%labels
\gantttitle{ระยะเวลาในการดำเนินการ}{18} \\
\gantttitle{มิย. 56}{2}
\gantttitle{กค. 56}{2} 
\gantttitle{สค. 56}{2} 
\gantttitle{กย. 56}{2} 
\gantttitle{ตค. 56}{2} 
\gantttitle{พย. 56}{2}
\gantttitle{ธค. 56}{2}
\gantttitle{มค. 57}{2}
\gantttitle{กพ. 57}{2} \\
%tasks
\ganttbar{1. ศึกษาข้อมูลพื้นฐานที่เกี่ยวข้อง}{1}{2} \\
\ganttbar{2. ศึกษาเทคโนโลยีที่จะนํามาใช้}{3}{4} \\
\ganttbar{3. วิเคราะห์และออกแบบ}{5}{8} \\
\ganttbar{4. พัฒนาระบบต้นแบบ}{7}{13} \\
\ganttbar{5. ทดสอบและตรวจสอบ}{9}{13} \\
\ganttbar{6. เรียบเรียงและจัดทําเอกสาร}{11}{18} \\
%\ganttbar[progress=33]{task 5}{20}{22} \\

%relations 
\ganttlink{elem0}{elem1}
\ganttlink{elem1}{elem2} 
\ganttlink{elem2}{elem3}
\ganttlink{elem3}{elem4} 
\ganttlink{elem4}{elem5}

\end{ganttchart}
\end{center}
%\caption{Gantt Chart}
%\end{figure}

\section{ประโยชน์โครงงานที่คาดว่าจะได้รับ}
ระบบต้นแบบที่พัฒนาขึ้นจะช่วยแก้ปัญหา ลดต้นทุนการผลิตบัตรสมาชิกของทางร้านค้า ลดการพกพาบัตรสมาชิกของลูกค้า และช่วยแก้ปัญหาในกรณีที่บัตรสมาชิกของลูกค้าเกิดศูนย์หาย โดยระบบต้นแบบดังกล่าวที่ถูกพัฒนาขึ้นสามารถนำไปใช้กับร้านค้า ร้านอาหาร ห้างสรรพสินค้าต่าง ๆ ได้

\section{รายการอ้างอิง}
\begin{enumerate}[label={[\arabic*]}]
  	\item \label{itm:shopping} Husni, E.; Purwantoro, S., "Shopping application system with Near Field Communication (NFC) based on Android," System Engineering and Technology (ICSET), 2012 International Conference on , vol., no., pp.1,6, 11-12 Sept. 2012
  	\item \label{itm:IDA-Pay} Mainetti, L.; Patrono, L.; Vergallo, R., "IDA-Pay: An innovative micro-payment system based on NFC technology for Android mobile devices," Software, Telecommunications and Computer Networks (SoftCOM), 2012 20th International Conference on , vol., no., pp.1,6, 11-13 Sept. 2012
	\item \label{itm:rpp-mobile} Resuscitating privacy-preserving mobile payment with customer in complete control
	\item \label{itm:prp-rfid} L. Catarinucci, S. Tedesco, D. De Donno, L. Tarricone: "Platform-Robust Passive UHF RFID Tags: a Case-Study in Robotics," Progress In Electromagnetics Research C, Vol. 30, 27-39, 2012.
	\item \label{itm:cicc} International Standard ISO/IEC 14443-1-2-3-4, Identification cards - Contactless integrated circuit cards - Proximity cards, 2008-07-15, ISO/IEC 2008, Switzerland.
\end{enumerate}

\end{document}  