% XeLaTeX can use any Mac OS X font. See the setromanfont command below.
% Input to XeLaTeX is full Unicode, so Unicode characters can be typed directly into the source.

% The next lines tell TeXShop to typeset with xelatex, and to open and save the source with Unicode encoding.

%!TEX TS-program = xelatex
%!TEX encoding = UTF-8 Unicode

%\documentclass[12pt]{article}
\documentclass[a4paper]{article}
\usepackage{geometry}                	% See geometry.pdf to learn the layout options. There are lots.
\geometry{letterpaper}                  % ... or a4paper or a5paper or ... 
%\geometry{landscape}                	% Activate for for rotated page geometry
%\usepackage[parfill]{parskip}    		% Activate to begin paragraphs with an empty line rather than an indent
\usepackage{graphicx}
\usepackage{amssymb}

%page dimension
%\hoffset = 0.5in
%\voffset = 0.5in

% Will Robertson's fontspec.sty can be used to simplify font choices.
% To experiment, open /Applications/Font Book to examine the fonts provided on Mac OS X,
% and change "Hoefler Text" to any of these choices.

\usepackage{fontspec,xltxtra,xunicode}
\defaultfontfeatures{Mapping=tex-text}
%\setromanfont[Mapping=tex-text]{Hoefler Text}
%\setsansfont[Scale=MatchLowercase,Mapping=tex-text]{Gill Sans}
%\setmonofont[Scale=MatchLowercase]{Andale Mono}
\setmainfont[Scale=1.1]{TH Sarabun New} 
\XeTeXlinebreaklocale 'th_TH'

% add a dot after the section number
\usepackage{titlesec}
\titlelabel{\thetitle.\quad}

% add tab
\newcommand{\tab}{\hspace{1.27cm}}

\reversemarginpar % Move the margin to the left of the page 
\newcommand{\MarginText}[1]{\marginpar{\raggedleft\itshape\small#1}} % New command defining the margin text style
%\newcommand{\MarginText}[1]{\marginpar{\raggedleft#1}} % New command defining the margin text style
\newcommand{\Description}[1]{\hangindent=2em\hangafter=0\noindent\raggedright\footnotesize{#1}\par\normalsize\vspace{1em}}
%\newcommand{\Description}[1]{\hangindent=1.27cm\hangafter=0\noindent\raggedright{#1}} % Define a command for descriptions of each entry - change spacing and font sizes here

\title{ข้อเสนอโครงงานมหาบัณฑิต (MASTER PROJECT PROPOSAL)}
\author{The Author}
%\date{}								% Activate to display a given date or no date

%Paragraph indent and break
\usepackage{indentfirst}
\setlength{\parindent}{1.27cm}
\setlength{\parskip}{1ex plus 0.5ex minus 0.2ex}
%\setlength{\parskip}{1cm plus4mm minus3mm}

% For many users, the previous commands will be enough.
% If you want to directly input Unicode, add an Input Menu or Keyboard to the menu bar 
% using the International Panel in System Preferences.
% Unicode must be typeset using a font containing the appropriate characters.
% Remove the comment signs below for examples.

% \newfontfamily{\A}{Geeza Pro}
% \newfontfamily{\H}[Scale=0.9]{Lucida Grande}
% \newfontfamily{\J}[Scale=0.85]{Osaka}
%----------------------------------------------------------------------------------------
\begin{document}
%\maketitle

\begin{center}
{\huge \bf ข้อเสนอโครงงานมหาบัณฑิต}
\end{center}

\noindent{{\LARGE \bf ชื่อหัวเรื่อง}}

%\Description{\MarginText{ภาษาไทยx}การพัฒนาระบบสำหรับการรวบรวมบัตรสมาชิก บนเทคโนโลยีเนียร์ฟิลด์คอมมูนิเคชันด้วยแอนดรอยด์แพลตฟอร์ม}
\Large{\noindent\hspace{0.7cm}\setlength{\tabcolsep}{15pt}
\begin{tabular}{l l}    
	ภาษาไทย 	& การพัฒนาระบบสำหรับการรวบรวมบัตรสมาชิก บนเทคโนโลยีเนียร์ฟิลด์ \\
				& คอมมูนิเคชันด้วยแอนดรอยด์แพลตฟอร์ม \\
	ภาษาอังกฤษ	& Developing a NFC Based Integrated Member Card System for \\
				& Mobile Devices Using the Android Platform \\    
\end{tabular}
}

% Here are some multilingual Unicode fonts: this is Arabic text: {\A السلام عليكم}, this is Hebrew: {\H שלום}, 
\section{ที่มาและความสำคัญของปัญหา}

เป็นที่ทราบกันในปัจจุบันนี้ว่า องค์กรส่วนใหญ่ต่างมุ่งไปที่ CRM (Customer relationship management) หรือพัฒนาด้านการจัดการลูกค้าสัมพันธ์ 
และหนึ่งในกระบวนการ CRM ก็คือการทําการตลาดเพื่อผูกสัมพันธ์กับลูกค้าระยะยาว เพื่อให้ลูกค้านั้นอยู่กับเราตลอดไป หรือที่เรียกกันว่า (CLV) Customer Lifetime Value การใช้บริการกลยุทธ์ Loyalty Program ถือเป็นโปรโมชั่นยอดนิยมแห่งยุค ธุรกิจหลากหลายรูปแบบต่าง
มุ่งประเด็นใช้สร้างสัมพันธ์กับลูกค้า ตั้งแต่สินค้า ภัตตาคาร ร้านอาหาร มือถือ บัตรเครดิต ห้างสรรพสินค้า สายการ
บิน โรงภาพยนตร์ และสถานบันเทิง ต่างแห่จัดโปรแกรมมัดใจลูกค้า ไม่ว่าจะเป็นสะสมคะแนน ศูนย์บริการรองรับ
ลูกค้าทางโทรศัพท์ และหนึ่งในนั้นคือ บัตรสมาชิกเพิ่มสิทธิประโยชน์ หรือส่วนลด เป็นต้น 

ข้อดีของการสมัครบัตรสมาชิกเพิ่มสิทธิประโยชน์ หรือส่วนลดก็คือ ช่วยให้เข้าใจความต้องการ และการตอบสนองของลูกค้าในสินค้าหรือบริการ สามารถสร้างผลกําไรในธุรกิจอย่างมีประสิทธิภาพ และสามารถดึงดูดลูกค้าให้กลับมาอีกครั้ง อีกทั้งช่วยเสริมให้โอกาสประสบความสําเร็จ รักษาลูกค้า เด่นชัด และได้ผลมากยิ่งขึ้น ด้วยเหตุนี้เององค์กรต่าง ๆ จึงมุ่งเน้นการทำบัตรสมาชิกเป็นจำนวนมาก แต่สิ่งที่ตามมากลับพบปัญหาว่ากลยุทธ์การทำตลาดดังกล่าวกลับไม่ได้ผลอย่างที่ควรจะเป็น อันเนื่องมาจากร้านทุกร้านต่างก็ทำบัตรสมาชิกเป็นของตัวเอง จึงไม่เกิดความแตกต่างในข้อได้เปรียบหรือเสียเปรียบ อีกทั้งยังเป็นภาระของต้นทุนที่ทุกร้านจะต้องจ่ายไปกับการทำบัตรสมาชิก ปัญหาที่เกิดขึ้นไม่ได้ส่งผลกระทบต่อองค์กรเพียงอย่างเดียว  ยังส่ง \newline ผลกระทบต่อลูกค้าด้วย ลูกค้าจะต้องพกพาบัตรสมาชิกของร้านค้าต่าง ๆ เป็นจำนวนมาก ยังไม่รวมถึงบัตรอื่น ๆ อย่างเช่น บัตรเครดิต บัตรประชาชน ใบขับขี่ ซึ่งไม่อำนวยความสะดวกในการพกพา

ปัจจุบันเทคโนโลยีก้าวเข้ามามีส่วนในชีวิตประจำวันของเรามากขึ้น ในแต่ละปีที่ผ่านไปจะเห็นว่ามีคนที่ใช้อุปกรณ์พกพากันมากขึ้น ส่งผลให้จำนวนการเติบโตของอุปกรณ์พกพาที่สูงขึ้นมาก ยิ่งไปกว่านั้นอุปกรณ์พกพาต่าง ๆ เริ่มจะที่ผนวกเทคโนโลยีแบบไร้สัมผัส ซึ่งจะช่วยรองรับการสื่อสารระหว่างเครื่องมืออิเล็กทรอนิกส์ในระยะใกล้ๆ ด้วยปัจจัยเหล่านี้เองถือเป็นโอกาสในการทรานส์ฟอร์มข้อมูลบัตรสมาชิกของแต่ละองค์กรต่าง ๆ ลงบนอุปกรณ์พกพา

โครงงานนี้มีจุดประสงค์เพื่อพัฒนาระบบสำหรับการรวบรวมบัตรสมาชิกบนเทคโนโลยีเนียร์ฟิลด์คอมมูนิเคชันด้วยแอนดรอยด์แพลตฟอร์ม
\section{ทฤษฏีที่เกี่ยวข้อง}

\section{งานวิจัยที่เกี่ยวข้อง}

\section{แนวคิดและวิธีการดำเนินงาน}

\section{วัตถุประสงค์}

\section{ขอบเขตการดำเนินงาน}

\section{ขั้นตอนการดำเนินงาน}

\section{ประโยชน์โครงงานที่คาดว่าจะได้รับ}

\section{รายการอ้างอิง}

\end{document}  